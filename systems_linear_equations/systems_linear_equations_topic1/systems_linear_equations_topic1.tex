% !TEX TS-program = xelatex
% !TEX encoding = UTF-8 Unicode

% Lecture Template for ME3001-001-Tristan Hill - Spring 2017 - Fall 2017 - Fall 2020
% Mechanical Engineering Analysis with MATLAB
% Module 3 - Systems of Linear Equations 
% Topic 1 - Linear Algebra Review  

\documentclass[fleqn]{beamer} % for presentation (has nav buttons at bottom)

\usepackage{/home/thill/Documents/lectures/analysis_lectures/analysis_lectures}

\newcommand{\MNUM}{3\hspace{2mm}} % Module number
\newcommand{\TNUM}{1\hspace{2mm}} % Topic number 
\newcommand{\moduletitle}{Systems of Linear Equations} % Titles and Stuff
\newcommand{\topictitle}{Linear Systems Review} 

\newcommand{\sectiontitleI}{What is a Linear Equation ?} % More Titles and Stuff
\newcommand{\sectiontitleII}{General Form of  A Linear System}
\newcommand{\sectiontitleIII}{The Geometrical Explanation}
\newcommand{\sectiontitleIV}{Examples in MATLAB}


\author{ME3001 - Mechanical Engineering Analysis} 
\title{Lecture Module - \moduletitle}
\date{Mechanical Engineering\vspc Tennessee Technological University}

\begin{document}
	
	\lstset{language=MATLAB,basicstyle=\ttfamily\small,showstringspaces=false}
	
	\frame{\titlepage \center\begin{framed}\Large \textbf{Topic \TNUM - \topictitle}\end{framed} \vspace{5mm}}
	
	% Section 0: Outline
	\frame{
		\large \textbf{Topic \TNUM - \topictitle} \vspace{3mm}\\
		
		\begin{itemize}
			
			\item \sectiontitleI    \vspc % Section I
			\item \sectiontitleII 	\vspc % Section II
			\item \sectiontitleIII 	\vspc %Section III
			\item \sectiontitleIV 	\vspc %Section IV
			
		\end{itemize}
		
	}

\section{\sectiontitleI}
\frame{
  \frametitle{\sectiontitleI}
  	
\textbf{What is a Linear Equation}

		\begin{itemize}
			\item ``A linear equation is an algebraic equation in which each term is either a constant or the product of a constant and a single variable'' - Wikipedia \vspace{3mm}\\
			
			\item slope intercept form	\vspace{3mm}\\
			
			\item does not contain \vspace{3mm}\\
		\end{itemize}	

 \textbf{ What is a System of Linear Equations?}
		\begin{itemize}
			\item multiple linear equations with... \vspace{3mm}\\
			\item also known as... \vspace{3mm}\\		
		\end{itemize}


}

\section{\sectiontitleII}

\frame{ \small
  \frametitle{\sectiontitleII}
 
	%\begin{multicols}{2}
		\begin{fleqn}
		
		The general system of linear equations  is shown with variables $x_{1,2,..,n}$ , coefficients $a_{11,12,..,nm}$, and knowns $b_{1,2,...,m}$ \vspace{1mm}\\		  
		\[a_{11} x_1 + a_{12} x_{2} + ... + a_{1n} x_n = b_1 \] 
		  \[a_{21} x_1 + a_{22} x_{2} + ... + a_{2n} x_n = b_2 \] 
		  \[\hspace{20mm}.\]		
		  \[a_{m1} x_1 + a_{m2} x_{2} + ... + a_{mn} x_n = b_m \]		
		  
		  
		  \[a_{11} x_1 + a_{12} x_{2} + ... + a_{1n} x_n = b_1 \] 
		  \[a_{21} x_1 + a_{22} x_{2} + ... + a_{2n} x_n = b_2 \] 	
		  \[...\]
		  \[a_{m1} x_1 + a_{m2} x_{2} + ... + a_{mn} x_n = b_m \]	
		  
		
		\end{fleqn}  
		 
		 The equations are cast into matrix form of the system. 	\\
		 	\scalebox{1.0}{\parbox{.5\linewidth}{
			\[ \left( \begin{array}{cccc}
			a_{11} & a_{12} & ...& a_{1n} \\
			a_{21} & a_{22} & ...& a_{2n} \\
			&.&&\\
			&.&&\\
			a_{m1} & a_{m2} & ...& a_{mn}\end{array} \right) \times \left[ \begin{array}{c}
			x_1 \\
			x_2 \\
			.\\
			.\\
			x_n \end{array} \right] = \left[ \begin{array}{c}
			b_1 \\
			b_2 \\
			.\\
			.\\
			b_m \end{array} \right]\] 
			}} \vspace{1mm}\\
			
	%\end{multicols}
}

\section{\sectiontitleIII}

\frame{ \small
  \frametitle{\sectiontitleIII}
  Consider the intersection of two Lines on the XY plane (2D).  \hspace{5mm} 
  

		\begin{itemize}
		
			\item Write an equation for each line.  $ax+by=c$ \vspace{3mm} \\
			 
		
			\item Organize the equations. \vspace{3mm} \\
		\end{itemize}
	
  }
  \frame{ \small
  \frametitle{\sectiontitleIII}
  Consider the intersection of two Lines on the XY plane (2D).  \hspace{5mm} 
  
		\begin{itemize}
		
			\item Cast the system into matrix form. \vspace{10mm}
			
			\item Solve the system. What exactly does this mean?\\
			\begin{itemize}
				\item \hspace{10mm} \\
				\item \hspace{10mm} \\
				\item \hspace{10mm} \\
			\end{itemize}
		
		\end{itemize}
	
  }

  \frame{ \small
  \frametitle{\sectiontitleIII}
  Repeat the exercise, and now consider the intersection of three planes in space (3D). What does the solution represent?  \hspace{3mm} \\ 
  

		\begin{itemize}
		
			\item Write an equation for each plane.  $ax+by+cz=d$ \vspace{3mm} \\
			 
		
			\item Organize the equations. \vspace{3mm} \\
		\end{itemize}
	
  }
  \frame{ \small
  \frametitle{\sectiontitleIII}

		\begin{itemize}
		
			\item Cast the system into matrix form. \vspace{10mm}
			
			\item Solve the system. What exactly does this mean?\\
			\begin{itemize}
				\item \hspace{10mm} \\
				\item \hspace{10mm} \\
				\item \hspace{10mm} \\
			\end{itemize}
		
		\end{itemize}
	
  }
\end{document}

%
%\begin{itemize}
%
%	\item  \textbf{\LARGE What is a Linear Equation}
%		\LARGE
%		\begin{itemize}
%			\item ``A linear equation is an algebraic equation in which each term is either a constant or the product of a constant and a single variable'' - Wikipedia \\ \vspace{10mm}
%			
%			\item slope intercept form	\\\vspace{20mm}
%			
%			\item does not contain \\\vspace{20mm}
%		\end{itemize}	
%
%	\item \textbf{\LARGE What is a System of Linear Equations?}
%		\begin{itemize}
%			\item multiple linear equations with... \\\vspace{20mm}
%			\item also known as... \\\vspace{20mm}				
%		\end{itemize}
%
%\newpage
%\item \textbf{\LARGE General Form of  A Linear System}
%	\begin{itemize}
%		\item The System of Linear Equations \\ \\
%		  \scalebox{1.5}{$a_{11} x_1 + a_{12} x_{2} + ... + a_{1n} x_n = b_1 $} \\
%		  \scalebox{1.5}{$a_{21} x_1 + a_{22} x_{2} + ... + a_{2n} x_n = b_2 $} \\
%		  \scalebox{1.5}{$\hspace{20mm}.$}\\
%		  \scalebox{1.5}{$\hspace{20mm}.$}\\
%		  \scalebox{1.5}{$\hspace{20mm}.$}\\		
%		  \scalebox{1.5}{$a_{n1} x_1 + a_{n2} x_{2} + ... + a_{nn} x_n = b_n $} \\\vspace{20mm}		
%		  
%		 
%		 \item The Matrix Form of the System 	\\\\
%		 	\scalebox{1.5}{\parbox{.5\linewidth}{
%			\[ \left( \begin{array}{cccc}
%			a_{11} & a_{12} & ...& a_{1n} \\
%			a_{21} & a_{22} & ...& a_{2n} \\
%			&.&&\\
%			&.&&\\
%			a_{n1} & a_{n2} & ...& a_{nn}\end{array} \right) \times \left[ \begin{array}{c}
%			x_1 \\
%			x_2 \\
%			.\\
%			.\\
%			x_n \end{array} \right] = \left[ \begin{array}{c}
%			b_1 \\
%			b_2 \\
%			.\\
%			.\\
%			b_n \end{array} \right]\] 
%			}} \vspace{10mm}\\
%			
%			 \item The Solution to the System of Equations	\\
%%		\newpage 
%%		 \item Matrix Multiplication  \\ \vspace{5mm}
%%		 
%%		 	 \scalebox{1.5}{ $C = A \times B \hspace{20mm} c_{ij}=\Sigma_{k=1}^n a_{ik}\times b_{kj}$} \\ \vspace{50mm}
%%		 
%%		 
%%		 \item Conformability \\ \vspace{30mm}
%%		 
%%		 
%%		 
%%	\end{itemize}
%%
%%\newpage
%%\item \textbf{\LARGE Basic Example}
%%
%%
%%\newpage
%%\item \textbf{\LARGE Basic Linear Algebra in MATLAB}
%%
%%	\begin{itemize}
%%
%%		\item \textbf{ \LARGE Useful Functions}\vspace{10mm}\\
%%
%%		\begin{multicols}{2}
%%			\scalebox{1.5}{{\fontfamily{qcr}\selectfont  \hspace{5mm} det()}} \VC \\
%%			\scalebox{1.5}{{\fontfamily{qcr}\selectfont  \hspace{5mm} rank()}} \VC\\
%%			\scalebox{1.5}{{\fontfamily{qcr}\selectfont  \hspace{5mm} inv()}} \VC\\
%%			\scalebox{1.5}{{\fontfamily{qcr}\selectfont  \hspace{5mm} `}} \VC\\
%%			
%%			\scalebox{1.5}{Determinant}\VC\\
%%			\scalebox{1.5}{Rank}\VC\\	
%%			\scalebox{1.5}{Inverse}\VC\\
%%			\scalebox{1.5}{Transpose}\VC\\
%%		
%%		\end{multicols}
%%
%
%\end{itemize}
%
%
%\newpage
%
%\item \textbf{\LARGE A Mechanical Engineering Example - Geometry}\\\\	
%		
%		As a group we are going to setup 2 small examples. \\\\
%		\begin{description}
%		\item [Example 1:] Intersection of 2 Lines.  \hspace{5mm} \scalebox{1.25}{ax+by=c} \\
%		\begin{enumerate}
%		
%			\item Write the individual equations. \vspace{100mm}
%		
%			\item Organize the equations.
%
%\newpage 
%		
%			\item Cast the system into matrix form. \vspace{140mm}
%			
%			\item Solve the system. \\
%			\begin{itemize}
%				\item \hspace{10mm} \\
%				\item \hspace{10mm} \\
%				\item \hspace{10mm} \\
%			\end{itemize}
%		
%		\end{enumerate}
%		
%		\newpage 
%		\item [Example 2:] Intersection of 3 Planes.  \scalebox{1.25}{ax+by+cz=d} \\\\
%		\begin{enumerate}
%		
%			\item Write the individual equations. \vspace{100mm}
%		
%			\item Organize the equations.
%
%\newpage 
%		
%			\item Cast the system into matrix form. \vspace{140mm}
%			
%			\item Solve the system. \\
%			\begin{itemize}
%				\item \hspace{10mm} \\
%				\item \hspace{10mm} \\
%				\item \hspace{10mm} \\
%			\end{itemize}
%		
%		\end{enumerate}
%		
%		
%
%		\end{description}
%
%		
%
%
%
%
%\newpage 
%
%	%\item \textbf{ \LARGE REMINDER - Homework 1 was due Friday, Turn it in late for reduced credit - ilearn and paper } \\
%	% \item \textbf{ \LARGE REMINDER - Homework 2 will be online tonight.} \\
%	%\item \textbf{ \LARGE REMINDER - MATLAB script from today's lecture will be posted on ilearn. } \\
%
%\end{itemize}
%
%
%	
%
%\end{document}
%
%
%
