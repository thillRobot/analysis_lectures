% Lecture Template for ME3001-001-Tristan Hill - Spring 2017 - Fall 2018
% 
% Mechanical Engineering Analysis with MATLAB
%
% Systems of Linear Equations - Lecture 1


% Document settings
\documentclass[11pt]{article}
\usepackage[margin=1in]{geometry}
\usepackage[pdftex]{graphicx}
\usepackage{multirow}
\usepackage{setspace}
\usepackage{hyperref}
\usepackage{color,soul}
\usepackage{fancyvrb}
\usepackage{framed}
\usepackage{wasysym}
\usepackage{multicol}

\pagestyle{plain}
\setlength\parindent{0pt}
\hypersetup{
    bookmarks=true,         % show bookmarks bar?
    unicode=false,          % non-Latin characters in Acrobat’s bookmarks
    pdftoolbar=true,        % show Acrobat’s toolbar?
    pdfmenubar=true,        % show Acrobat’s menu?
    pdffitwindow=false,     % window fit to page when opened
    pdfstartview={FitH},    % fits the width of the page to the window
    pdftitle={My title},    % title
    pdfauthor={Author},     % author
    pdfsubject={Subject},   % subject of the document
    pdfcreator={Creator},   % creator of the document
    pdfproducer={Producer}, % producer of the document
    pdfkeywords={keyword1} {key2} {key3}, % list of keywords
    pdfnewwindow=true,      % links in new window
    colorlinks=true,       % false: boxed links; true: colored links
    linkcolor=red,          % color of internal links (change box color with linkbordercolor)
    citecolor=green,        % color of links to bibliography
    filecolor=magenta,      % color of file links
    urlcolor=blue           % color of external links
}

% assignment number 

\definecolor{mygreen}{rgb}{0, .39, 0}

%\definecolor{dred}{#8B0000}
% [153,50,204] - dark orchid
\definecolor{mypurple}{rgb}{0.6,0.1961,0.8}
%[139,69,19] - saddle brown
\definecolor{mybrown}{rgb}{0.5451,0.2706,0.0745}

\definecolor{mygray}{rgb}{.6, .6, .6}

\setulcolor{red} 
\setstcolor{green} 
\sethlcolor{mygray} 

\newcommand{\VA}{\vspace{2mm}}
\newcommand{\VB}{\vspace{5mm}} 
\newcommand{\VC}{\vspace{30mm}} 
 
\newcommand{\R}{\color{red}}
\newcommand{\B}{\color{blue}}
\newcommand{\K}{\color{black}}
\newcommand{\G}{\color{mygreen}}
\newcommand{\PR}{\color{mypurple}}

\newcommand{\NUM}{1} 
\newcommand{\VSpaceSize}{2mm} 
\newcommand{\HSpaceSize}{2mm} 

\begin{document}

\textbf{ \LARGE ME 3001 Lecture - Systems of Linear Equations} \\\\
\textbf{ \LARGE A Brief Review of Linear Algebra in MATLAB} \\

\begin{itemize}

	\item  \textbf{\LARGE What is a Linear Equation}
		\LARGE
		\begin{itemize}
			\item ``A linear equation is an algebraic equation in which each term is either a constant or the product of a constant and a single variable'' - Wikipedia \\ \vspace{10mm}
			
			\item slope intercept form	\\\vspace{20mm}
			
			\item does not contain \\\vspace{20mm}
		\end{itemize}	

	\item \textbf{\LARGE What is a System of Linear Equations?}
		\begin{itemize}
			\item multiple linear equations with... \\\vspace{20mm}
			\item also known as... \\\vspace{20mm}				
		\end{itemize}

\newpage
\item \textbf{\LARGE General Form of  A Linear System}
	\begin{itemize}
		\item The System of Linear Equations \\ \\
		  \scalebox{1.5}{$a_{11} x_1 + a_{12} x_{2} + ... + a_{1n} x_n = b_1 $} \\
		  \scalebox{1.5}{$a_{21} x_1 + a_{22} x_{2} + ... + a_{2n} x_n = b_2 $} \\
		  \scalebox{1.5}{$\hspace{20mm}.$}\\
		  \scalebox{1.5}{$\hspace{20mm}.$}\\
		  \scalebox{1.5}{$\hspace{20mm}.$}\\		
		  \scalebox{1.5}{$a_{n1} x_1 + a_{n2} x_{2} + ... + a_{nn} x_n = b_n $} \\\vspace{20mm}		
		  
		 
		 \item The Matrix Form of the System 	\\\\
		 	\scalebox{1.5}{\parbox{.5\linewidth}{
			\[ \left( \begin{array}{cccc}
			a_{11} & a_{12} & ...& a_{1n} \\
			a_{21} & a_{22} & ...& a_{2n} \\
			&.&&\\
			&.&&\\
			a_{n1} & a_{n2} & ...& a_{nn}\end{array} \right) \times \left[ \begin{array}{c}
			x_1 \\
			x_2 \\
			.\\
			.\\
			x_n \end{array} \right] = \left[ \begin{array}{c}
			b_1 \\
			b_2 \\
			.\\
			.\\
			b_n \end{array} \right]\] 
			}} \vspace{10mm}\\
			
			 \item The Solution to the System of Equations	\\
%		\newpage 
%		 \item Matrix Multiplication  \\ \vspace{5mm}
%		 
%		 	 \scalebox{1.5}{ $C = A \times B \hspace{20mm} c_{ij}=\Sigma_{k=1}^n a_{ik}\times b_{kj}$} \\ \vspace{50mm}
%		 
%		 
%		 \item Conformability \\ \vspace{30mm}
%		 
%		 
%		 
%	\end{itemize}
%
%\newpage
%\item \textbf{\LARGE Basic Example}
%
%
%\newpage
%\item \textbf{\LARGE Basic Linear Algebra in MATLAB}
%
%	\begin{itemize}
%
%		\item \textbf{ \LARGE Useful Functions}\vspace{10mm}\\
%
%		\begin{multicols}{2}
%			\scalebox{1.5}{{\fontfamily{qcr}\selectfont  \hspace{5mm} det()}} \VC \\
%			\scalebox{1.5}{{\fontfamily{qcr}\selectfont  \hspace{5mm} rank()}} \VC\\
%			\scalebox{1.5}{{\fontfamily{qcr}\selectfont  \hspace{5mm} inv()}} \VC\\
%			\scalebox{1.5}{{\fontfamily{qcr}\selectfont  \hspace{5mm} `}} \VC\\
%			
%			\scalebox{1.5}{Determinant}\VC\\
%			\scalebox{1.5}{Rank}\VC\\	
%			\scalebox{1.5}{Inverse}\VC\\
%			\scalebox{1.5}{Transpose}\VC\\
%		
%		\end{multicols}
%

\end{itemize}


\newpage

\item \textbf{\LARGE A Mechanical Engineering Example - Geometry}\\\\	
		
		As a group we are going to setup 2 small examples. \\\\
		\begin{description}
		\item [Example 1:] Intersection of 2 Lines.  \hspace{5mm} \scalebox{1.25}{ax+by=c} \\
		\begin{enumerate}
		
			\item Write the individual equations. \vspace{100mm}
		
			\item Organize the equations.

\newpage 
		
			\item Cast the system into matrix form. \vspace{140mm}
			
			\item Solve the system. \\
			\begin{itemize}
				\item \hspace{10mm} \\
				\item \hspace{10mm} \\
				\item \hspace{10mm} \\
			\end{itemize}
		
		\end{enumerate}
		
		\newpage 
		\item [Example 2:] Intersection of 3 Planes.  \scalebox{1.25}{ax+by+cz=d} \\\\
		\begin{enumerate}
		
			\item Write the individual equations. \vspace{100mm}
		
			\item Organize the equations.

\newpage 
		
			\item Cast the system into matrix form. \vspace{140mm}
			
			\item Solve the system. \\
			\begin{itemize}
				\item \hspace{10mm} \\
				\item \hspace{10mm} \\
				\item \hspace{10mm} \\
			\end{itemize}
		
		\end{enumerate}
		
		

		\end{description}

		




\newpage 

	%\item \textbf{ \LARGE REMINDER - Homework 1 was due Friday, Turn it in late for reduced credit - ilearn and paper } \\
	% \item \textbf{ \LARGE REMINDER - Homework 2 will be online tonight.} \\
	%\item \textbf{ \LARGE REMINDER - MATLAB script from today's lecture will be posted on ilearn. } \\

\end{itemize}


	

\end{document}



