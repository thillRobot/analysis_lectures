% !TEX TS-program = xelatex
% !TEX encoding = UTF-8 Unicode

% In Class Activity for ME3001- Tristan Hill 
% Spring 2017 - Fall 2017 - Fall 2020 - Fall 2021 - Spring 2024
% Mechanical Engineering Analysis with MATLAB
% Activity 7 -(In-Class)- Analytical and Numerical Solutions to ODEs 

\documentclass[12pt]{article}
\usepackage{/home/thill/courses/analysis_lectures/analysis_activities}
%\usepackage{/home/tntech.edu/thill/courses/analysis_lectures/analysis_activities}

% Title and Misc
\newcommand{\COURNAME}{ME 3001-002}
\newcommand{\CURRTERM}{Spring 2024} %Current Term
\newcommand{\MNUM}{1} %Module Number
\newcommand{\ANUM}{7} %Activity Number
\newcommand{\moduletitle}{Analytical and Numerical Solutions to ODEs}
\newcommand{\activitytitle}{Linear Pendulum} %Module Name
\pagestyle{myheadings}
%\markright{{\large ME4140 - ROS Workshop - \CURRTERM}}

\textwidth=7.0in
\topmargin=-0.6in
\leftmargin=0.5in
\textheight=9.25in
\hoffset=-0.5in
\footskip=0.2in

\begin{document}

\thispagestyle{plain}

\begin{center}
   {\bf \Large In-Class Activity\hspc\ANUM\hspc - \activitytitle}\vspace{3mm}\\
   {\bf \large \COURNAME - Mechanical Engineering Analysis - \CURRTERM} \vspace{5mm}\\
\end{center}

\begin{description}


\item[\textbf{\underline{Learning Objectives:}}] \hfill \vspace{0mm}

\begin{itemize}
	\item Demonstrate solving an ordinary differential equation with an analytical method
	\item Demonstrate approximating the solution to an ordinary differential equation with a numerical method
	\item Practice using plotting in MATLAB to to display solution results
\end{itemize}

\item[\textbf{\underline{Peer Collaboration:}}] \hfill \vspace{0mm}
	
	{\bf This is an individual assignment}, but you are encouraged to discuss the problem with your peers. You must write your own program and submit as an individual, but you can share ideas about the algorithm with your peers and the instructor.

\item[\textbf{\underline{Overview:}}] \hfill \vspace{0mm}


	The non-linear pendulum model exposed to a cose is given above. 	

   \[\frac{d^2\theta}{dt^2}+\frac{g}{l}\sin{\theta}=f_{wind}\hspace{10mm} \] 
	\[ l=0.35 \hspc(m), f_{wind} = 5 \cos{\theta} \hspc(N) \]

	The model can be linearized with the small angle indentity with results in the following equation.
 	
 	\[ \frac{d^2\theta}{dt^2}+\frac{g}{l}\theta=f_{wind} \]

 	The displacement of the pendulum can be found as the solution to the ODE as an initial value problem. Initial conditions are given.
 	\[ \theta(0)=25^\circ, \hspcc\frac{d\theta}{dt}|_{t=0}=0 \]



\item[\textbf{\underline{Activity:}}] \hfill \vspace{0mm}

\begin{enumerate}	
	
	\item Find the displacement of the pendulum by solving for the linearized pendulum ODE given the initial conditions using an analytical method. Clearly show all of the steps required find solution. 

	\item Write a MATLAB program use Euler's Method or alternate numerical method to approximate the solution condidering the same initial conditions. The program should plot the analyical and numerical results in the same figure for comparison. Do the results agree? If not suggest sources of error.
	
	\item Write a brief description of how your program works to solve the problem. This can be a few sentences or a bulleted list. 

\end{enumerate}

\newpage	



\item[\textbf{\underline{Submission:}}] \hfill \vspace{0mm}

	\begin{itemize}

		\item Submit the MATLAB program as a .m file. The program should run free from errors and produce the results described in the assignment including any figure. 

		\item Include answers to any discussion questions in a seperate document or the assignment submission textbox. 

	\end{itemize}		


		%Submit the most complete version of {\bf \BL<USERNAME>\BK\_activity\ANUM.pdf} \\and {\bf \BL<USERNAME>\BK\_activity6.m } to the Activity \ANUM \hspace{1mm} folder before the posted due date.

\end{description}
\end{document}
