 
% ME3001   
% Tristan Hill - Spring 2024 
% Finding Roots of Non-Linear Functions 
% The Newton-Raphson method basic example


% Document setting
\documentclass[11pt]{article}
\usepackage[margin=1in]{geometry}
\usepackage[pdftex]{graphicx}
\usepackage{multirow}
\usepackage{setspace}
\usepackage{hyperref}
\usepackage{color,soul}
\usepackage{fancyvrb}
\usepackage{framed}
\usepackage{wasysym}
\usepackage{multicol}

\pagestyle{plain}
\setlength\parindent{0pt}


% assignment number 
\newcommand{\NUM}{1} 
\newcommand{\HSPC}{42mm} 
\newcommand{\HHSPC}{33mm} 
\newcommand{\VSpaceSize}{2mm} 
\newcommand{\HSpaceSize}{2mm} 

\definecolor{mygray}{rgb}{.6, .6, .6}

% [153,50,204] - dark orchid
\definecolor{mypurple}{rgb}{0.6,0.1961,0.8}
%[139,69,19] - saddle brown
\definecolor{mybrown}{rgb}{0.5451,0.2706,0.0745}


\begin{document}

	\textbf{\LARGE ME3001 - Spring 2024} \\\\
	\textbf{\LARGE Weekly Activity \NUM:  Roots of Non-Linear Equations}\\\\
	\textbf{\LARGE The Netwon-Raphson Method} \\
	
	\begin{description}
        \vspace{5mm}
    \item [\textbf{ \Large Overview}] \textbf{ \Large :}\\

    In this activity, you will practice using the Newton-Raphson method to approximate the roots of a polynomial. MATLAB will be used to implement the solution approach.

    \item [\textbf{ \large Assignment}] \textbf{ \Large :}\\

  \begin{enumerate}

    \item Solve the example problem shown in Non-Linear equations notes using Newton-Raphson in MATLAB. The goal is to verify the analytical solutions (aka roots) found in class to the polynomial shown below. 

    \[y(x) = x^2 +2x - 10\]

  \end{enumerate}

  \item [\textbf{ \large Deliverables}] \textbf{ \Large :}\\
  \begin{itemize} 
   
    \item Write a MATLAB program that uses the Newton-Raphson method to find both roots of the equation. One option would be to use the same program twice with different starting guesses. Submit the .m file(s) used and document any example code that you used or learned from during the exercise.
\vspace*{3mm}
    \item For each root show the starting guess used and the final approximate root (x value) and function value (y value) at the approximate root. The results can be copied or typed into the text for the assignment or included as a comment in the code.

  \end{itemize}
  \end{description}
 
\end{document}



