% ME 3001 - Tristan Hill - Spring 2017
% Homework 7

% The 1D Heat Equation
% Solved With Finite Difference

% Document settings
\documentclass[11pt]{article}
\usepackage[margin=1in]{geometry}
\usepackage[pdftex]{graphicx}
\usepackage{multirow}
\usepackage{setspace}
\usepackage{hyperref}
\usepackage{color,soul}
\usepackage{fancyvrb}
\usepackage{framed}
\usepackage{wasysym}
\usepackage{amssymb}

\pagestyle{plain}
\setlength\parindent{0pt}
\hypersetup{
    bookmarks=true,         % show bookmarks bar?
    unicode=false,          % non-Latin characters in Acrobat’s bookmarks
    pdftoolbar=true,        % show Acrobat’s toolbar?
    pdfmenubar=true,        % show Acrobat’s menu?
    pdffitwindow=false,     % window fit to page when opened
    pdfstartview={FitH},    % fits the width of the page to the window
    pdftitle={My title},    % title
    pdfauthor={Author},     % author
    pdfsubject={Subject},   % subject of the document
    pdfcreator={Creator},   % creator of the document
    pdfproducer={Producer}, % producer of the document
    pdfkeywords={keyword1} {key2} {key3}, % list of keywords
    pdfnewwindow=true,      % links in new window
    colorlinks=true,       % false: boxed links; true: colored links
    linkcolor=red,          % color of internal links (change box color with linkbordercolor)
    citecolor=green,        % color of links to bibliography
    filecolor=magenta,      % color of file links
    urlcolor=blue           % color of external links
}

% assignment number 
\newcommand{\NUM}{6} 

\definecolor{mygray}{rgb}{.6, .6, .6}

\setulcolor{red} 
\setstcolor{green} 
\sethlcolor{mygray} 

\begin{document}

	\textbf{\LARGE ME 3001-001, Fall 2019} \\\\
	\textbf{\LARGE Activity : FDM - 1D Vibration - Wave Equation} \\\\

	The 1D vibration of an elastic member is governed by the following partial differential equation known as the `1D wave equation'. \\

	\scalebox{1.5}{$\frac{\partial^2 u}{\partial t^2} = c^2\frac{\partial^2 u}{\partial x^2}$} \\

	It is important to define the boundary conditions and the inital conditions carefully. \vspace{50mm}\\


	With the FDEs we can transform the PDE into a new equation in terms of $u_i^{j+1}, u_{i-1}^{j}, u_{i}^{j},u_{i+1}^{j},u_i^{j-1}$ \vspace{30mm} \\\\
	
	Equation 2 can be used explicitly but we have to start carefully.

		


		
\end{document}



