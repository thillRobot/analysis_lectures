% Lecture Template for ME3001-001-Tristan Hill - Spring 2017
% 
% Mechanical Engineering Analysis with MATLAB
%
% Systems of Linear Equations - Lecture 2
%


% Document settings
\documentclass[11pt]{article}
\usepackage[margin=1in]{geometry}
\usepackage[pdftex]{graphicx}
\usepackage{multirow}
\usepackage{setspace}
\usepackage{hyperref}
\usepackage{color,soul}
\usepackage{fancyvrb}
\usepackage{framed}
\usepackage{wasysym}
\usepackage{multicol}
\usepackage{amssymb}

\pagestyle{plain}
\setlength\parindent{0pt}
\hypersetup{
    bookmarks=true,         % show bookmarks bar?
    unicode=false,          % non-Latin characters in Acrobat’s bookmarks
    pdftoolbar=true,        % show Acrobat’s toolbar?
    pdfmenubar=true,        % show Acrobat’s menu?
    pdffitwindow=false,     % window fit to page when opened
    pdfstartview={FitH},    % fits the width of the page to the window
    pdftitle={My title},    % title
    pdfauthor={Author},     % author
    pdfsubject={Subject},   % subject of the document
    pdfcreator={Creator},   % creator of the document
    pdfproducer={Producer}, % producer of the document
    pdfkeywords={keyword1} {key2} {key3}, % list of keywords
    pdfnewwindow=true,      % links in new window
    colorlinks=true,       % false: boxed links; true: colored links
    linkcolor=red,          % color of internal links (change box color with linkbordercolor)
    citecolor=green,        % color of links to bibliography
    filecolor=magenta,      % color of file links
    urlcolor=blue           % color of external links
}

% assignment number 
\newcommand{\NUM}{3} 
\newcommand{\VSpaceSize}{2mm} 
\newcommand{\HSpaceSize}{2mm} 

\definecolor{mygray}{rgb}{.6, .6, .6}
\definecolor{mypurple}{rgb}{0.6,0.1961,0.8}
\definecolor{mygreen}{rgb}{0.1333 ,  0.5451,    0.1333}
\definecolor{mypink}{rgb}{0.1333 ,  0.5451,    0.1333}
\setulcolor{red} 
\setstcolor{green} 
\sethlcolor{mygray} 

\begin{document}

\textbf{ \LARGE ME 3001 Lecture, Systems of Linear Equations} \\\\
\textbf{ \LARGE The Gaussian Elimination Algorithm} \\


 \renewcommand\labelitemi{\textbullet}
 \renewcommand\labelitemii{\textendash}
 \renewcommand\labelitemiii{\textasteriskcentered}
 \renewcommand\labelitemiv{\textperiodcentered}

	
	\large
	\begin{itemize}
		
		
		\item {\bf The Forward Eimination Algorithm:}  \\

		\scalebox{1.5}{ {\it for} \color{mypurple}k \color{black} {\it from} 1 {\it to} \color{mygreen}n\color{black}-1} \\\\
	
		\hspace{10mm} \scalebox{1.5}{{\it for} \color{blue}i \color{black} {\it from} \color{mypurple}k\color{black}+1 {\it to} \color{mygreen}n\color{black}} \\\\
		
		\hspace{20mm} \scalebox{1.5}{fact$=a_{\color{blue}i\color{black},\color{mypurple}k}/a_{\color{mypurple}k\color{black},\color{mypurple}k\color{black}}$} \\

		\hspace{20mm} \scalebox{1.5}{{\it for} \color{red}j \color{black} {\it from} \color{mypurple} k \color{black}  {\it to} \color{mygreen}n\color{black}}\\\\

		\hspace{30mm} \scalebox{1.5}{$a_{\color{blue}i\color{black},\color{red}j}=a_{\color{blue}i\color{black},\color{red}j}-$fact$\times a_{\color{mypurple}k\color{black},\color{red}j}$}\\\\
\hspace*{20mm}\scalebox{1.5}{{\it end}}\\\\
\hspace*{20mm}\scalebox{1.5}{$b_{\color{blue}i\color{black}}=b_{\color{blue}i\color{black}}-$fact$\times b_{\color{mypurple}k\color{black}}$}\\\\
\hspace*{10mm}\scalebox{1.5}{{\it end}}\\ \\	
\scalebox{1.5}{{\it end}}\\ 	 
		%\hspace*{20mm} \scalebox{1.5}{\color{blue} end \color{black} }\\
		%\hspace{0mm} \scalebox{1.5}{\color{blue} end \color{black}}\\ \\
		%\hspace{0mm} \scalebox{1.5}{\color{blue} end \color{black}}\\

		
		\item {\bf The Backwards Substitution Algorithm:}  \\
		
		\scalebox{1.5}{$x_{\color{mygreen}n\color{black}}=b_{\color{mygreen}n\color{black}}/a_{\color{mygreen}n\color{black},\color{mygreen}n\color{black}}$}\\\\
		\scalebox{1.5}{{\it for} \color{blue}i \color{black} {\it from} \color{mygreen}n\color{black}-1 {\it to} 1} \\\\
	
		\hspace{10mm}\scalebox{1.5}{$x_{\color{blue}i}=(b_{\color{blue}i}-$$\sum\limits^n_{\color{red}j\color{black}=\color{blue}i\color{black}+1}$$\left( a_{\color{blue}i\color{black},\color{red}j}x_{\color{red}j}\right)) /a_{\color{blue}i\color{black},\color{blue}i\color{black}}$}	\\\\
		\scalebox{1.5}{{\it end}}\\ 


		
		\end{itemize}
	

\end{document}



