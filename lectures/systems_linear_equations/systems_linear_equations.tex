% Lecture Template for ME3001 - Mechanical Engineering Analysis - Tennessee Technological University
% Spring 2024 - condensing and streamlining lectures by combining topics into a single PDF under the module name
% this will simplify file and link management as well as make lectures easier to use in class
% - added image/ to clean directory and reduce redundancy, specific to module for now  

% Module Name: - Systems of Linear Equations
% Topic 1 - 
% Topic 2 - 
% Topic 3 - 
% Topic 4 - 

\documentclass[fleqn]{beamer} % for presentation (has nav buttons at bottom)

%\usepackage{/home/tntech.edu/thill/courses/analysis_lectures/lectures/analysis_lectures}
%\usepackage{/home/thill/courses/analysis_lectures/lectures/analysis_lectures}
\usepackage{/mnt/c/Users/thill/Documents/courses/analysis_lectures/lectures/analysis_lectures}

\author{ME3001 - Mechanical Engineering Analysis}

\newcommand{\MNUM}{3\hspace{2mm}} % module number 
\newcommand{\moduletitle}{Systems of Linear Equations}

\newcommand{\sectionItitle}{Linear Systems Review}
\newcommand{\sectionIItitle}{}
\newcommand{\sectionIIItitle}{}
\newcommand{\sectionIVtitle}{}

\newcommand{\sectionIsubsectionItitle}{What is a Linear Equation ?}
\newcommand{\sectionIsubsectionIItitle}{General Form of  A Linear System}
\newcommand{\sectionIsubsectionIIItitle}{The Geometrical Explanation}
\newcommand{\sectionIsubsectionIVtitle}{Examples in MATLAB}

\newcommand{\sectionIIsubsectionItitle}{}
\newcommand{\sectionIIsubsectionIItitle}{}
\newcommand{\sectionIIsubsectionIIItitle}{}
\newcommand{\sectionIIsubsectionIVtitle}{}
\newcommand{\sectionIIsubsectionVtitle}{}
\newcommand{\sectionIIsubsectionVItitle}{}

\newcommand{\sectionIIIsubsectionItitle}{}
\newcommand{\sectionIIIsubsectionIItitle}{}
\newcommand{\sectionIIIsubsectionIIItitle}{}
\newcommand{\sectionIIIsubsectionIVtitle}{}

\newcommand{\sectionIVsubsectionItitle}{}
\newcommand{\sectionIVsubsectionIItitle}{}
\newcommand{\sectionIVsubsectionIIItitle}{}
\newcommand{\sectionIVsubsectionIVtitle}{}

% custom box
\newsavebox{\mybox}

\title{Lecture Module - \moduletitle}

\date{Mechanical Engineering\vspc Tennessee Technological University}

\begin{document}

	\lstset{language=MATLAB,basicstyle=\ttfamily\small,showstringspaces=false}

	\frame{\titlepage \center\begin{framed}\Large \textbf{Module \MNUM - \moduletitle}\end{framed} \vspace{5mm}}

	% Module Outline
	\begin{frame} 
		\large \textbf{Module \MNUM - \moduletitle} \vspace{3mm}\\

		\begin{itemize}
			\item Topic 1 - \hyperlink{sectionI}{\sectionItitle} \vspc % section I
			\item Topic 2 - \hyperlink{sectionII}{\sectionIItitle} \vspc % section II
			\item Topic 3 - \hyperlink{sectionIII}{\sectionIIItitle} \vspc % section III
		\end{itemize}

	\end{frame}

	% section I
	\section{\sectionItitle}\label{sectionI}

		% section I Outline
		\begin{frame} 
			\large \textbf{Topic 1 - \sectionItitle} \vspace{3mm}\\

			\begin{itemize}
				\item \hyperlink{sectionIsubsectionI}{\sectionIsubsectionItitle} \vspc %  section I subsection I
				\item \hyperlink{sectionIsubsectionII}{\sectionIsubsectionIItitle} \vspc % section I subsection II
				\item \hyperlink{sectionIsubsectionIII}{\sectionIsubsectionIIItitle} \vspc % section I subsection III
				\item \hyperlink{sectionIsubsectionIV}{\sectionIsubsectionIVtitle} \vspc % section I subsection IV
			\end{itemize}
		\end{frame}
		
		% section I subsection I 
		\subsection{\sectionIsubsectionItitle}\label{sectionIsubsectionI}

			\begin{frame}
				\frametitle{\sectionIsubsectionItitle}
				\bigskip

				\textbf{What is a Linear Equation}

				\begin{itemize}
					\item ``A linear equation is an algebraic equation in which each term is either a constant or the product of a constant and a single variable'' - Wikipedia \vspace{3mm}\\
					\item slope intercept form	\vspace{3mm}\\
					\item does not contain \vspace{3mm}\\
				\end{itemize}	

				\textbf{ What is a System of Linear Equations?}
				\begin{itemize}
					\item multiple linear equations with... \vspace{3mm}\\
					\item also known as... \vspace{3mm}\\		
				\end{itemize}

				\btVFill
			\end{frame}

		% section I subsection II
		\subsection{\sectionIsubsectionIItitle}\label{sectionIsubsectionII}

			\begin{frame}
				\frametitle{\sectionIsubsectionIItitle} \small
				\bigskip

				\begin{fleqn}
		
					The general system of linear equations  is shown with variables $x_{1,2,..,n}$ , coefficients $a_{11,12,..,nm}$, and knowns $b_{1,2,...,m}$ \vspace{1mm}\\		  
					
					\[a_{11} x_1 + a_{12} x_{2} + ... + a_{1n} x_n = b_1 \] 
					\[a_{21} x_1 + a_{22} x_{2} + ... + a_{2n} x_n = b_2 \] 
					\[\hspace{20mm}.\]		
					\[a_{m1} x_1 + a_{m2} x_{2} + ... + a_{mn} x_n = b_m \]			

				\end{fleqn}  

				The equations are cast into matrix form of the system. 	\\
				\begin{fleqn}

					\[ \left( \begin{array}{cccc}
					a_{11} & a_{12} & ...& a_{1n} \\
					a_{21} & a_{22} & ...& a_{2n} \\
					&.&&\\
					&.&&\\
					a_{m1} & a_{m2} & ...& a_{mn}\end{array} \right) \times \left[ \begin{array}{c}
					x_1 \\
					x_2 \\
					.\\
					.\\
					x_n \end{array} \right] = \left[ \begin{array}{c}
					b_1 \\
					b_2 \\
					.\\
					.\\
					b_m \end{array} \right]\] 

				\end{fleqn}

				\btVFill
			\end{frame}

				%\btVFill
			\begin{frame}
				\frametitle{\sectionIsubsectionIItitle} \small
				\bigskip

				To verify the matrix form $[A]\{x\}=\{b\}$ is correct, use matric multiplication and the result will match the individual equations.	

				\begin{fleqn}

					\[ \left( \begin{array}{cccc}
					a_{11} & a_{12} & ...& a_{1n} \\
					a_{21} & a_{22} & ...& a_{2n} \\
					&.&&\\
					&.&&\\
					a_{m1} & a_{m2} & ...& a_{mn}\end{array} \right) \times \left[ \begin{array}{c}
					x_1 \\
					x_2 \\
					.\\
					.\\
					x_n \end{array} \right] = \left[ \begin{array}{c}
					b_1 \\
					b_2 \\
					.\\
					.\\
					b_m \end{array} \right]\] 

				\end{fleqn}


				\btVFill
			\end{frame}

		% section I subsection III
		\subsection{\sectionIsubsectionIIItitle}\label{sectionIsubsectionIII}
			\begin{frame} 
				\frametitle{\sectionIsubsectionIIItitle}
				\bigskip

				 Consider the intersection of two Lines on the XY plane (2D).  \hspace{5mm} 

					\begin{itemize}
	
						\item Write an equation for each line.  $ax+by=c$ \vspace{3mm} \\
	
						\item Organize the equations. \vspace{3mm} \\
					
					\end{itemize}

				\btVFill
			\end{frame}	

			\begin{frame} 
				\frametitle{\sectionIsubsectionIIItitle}
				\bigskip

			 	Consider the intersection of two Lines on the XY plane (2D).  \hspace{5mm} 
  
				\begin{itemize}
		
					\item Cast the system into matrix form. \vspace{10mm}
			
					\item Solve the system. What exactly does this mean?\\
				
					\begin{itemize}
							\item \hspace{10mm} \\
							\item \hspace{10mm} \\
							\item \hspace{10mm} \\
					\end{itemize}
			
				\end{itemize}
	
				\btVFill
			\end{frame}	

			\begin{frame} 
				\frametitle{\sectionIsubsectionIIItitle}
				\bigskip

		 		Repeat the exercise, and now consider the intersection of three planes in space (3D). What does the solution represent?  \hspace{3mm} \\ 
  
				\begin{itemize}
		
					\item Write an equation for each plane.  $ax+by+cz=d$ \vspace{3mm} \\
		
					\item Organize the equations. \vspace{3mm} \\
				\end{itemize}
				
				\btVFill
			\end{frame}	

			\begin{frame} 
				\frametitle{\sectionIsubsectionIIItitle}
				\bigskip
	
				\begin{itemize}
		
					\item Cast the system into matrix form. \vspace{10mm}
			
					\item Solve the system. What exactly does this mean?\\
					\begin{itemize}
						\item \hspace{10mm} \\
						\item \hspace{10mm} \\
						\item \hspace{10mm} \\
					\end{itemize}
		
				\end{itemize}
			
				\btVFill
			\end{frame}	



		% section I subsection IV
		\subsection{\sectionIsubsectionIVtitle}\label{sectionIsubsectionIV}	

			\begin{frame}
				\frametitle{\sectionIsubsectionIVtitle}
				\bigskip


				\btVFill
			\end{frame}
	
	% Section II
	\section{\sectionIItitle}\label{sectionII}

		% section II Outline
		\begin{frame}
			\large \textbf{Topic 2 - \sectionIItitle} \vspace{3mm}\\

			\begin{itemize}
				\item \hyperlink{sectionIIsubsectionI}{\sectionIIsubsectionItitle} \vspc %  section II subsection I
				\item \hyperlink{sectionIIsubsectionII}{\sectionIIsubsectionIItitle} \vspc % section II subsection II
				\item \hyperlink{sectionIIsubsectionIII}{\sectionIIsubsectionIIItitle} \vspc % section II subsection III
				\item \hyperlink{sectionIIsubsectionIV}{\sectionIIsubsectionIVtitle} \vspc % section II subsection IV
				\item \hyperlink{sectionIIsubsectionV}{\sectionIIsubsectionVtitle} \vspc % section II subsection V
				\item \hyperlink{sectionIIsubsectionVI}{\sectionIIsubsectionVItitle} \vspc % section II subsection VI
			\end{itemize}

		\end{frame}

		% section II subsection I
		\subsection{\sectionIIsubsectionItitle}\label{sectionIIsubsectionI}

			\begin{frame}[label=sectionIIsubsectionI]
				\frametitle{\sectionIIsubsectionItitle}
				\bigskip


				\btVFill
			\end{frame}

			\begin{frame}[label=sectionIIsubsectionI]
				\frametitle{\sectionIIsubsectionItitle}
				\bigskip


				\btVFill
			\end{frame}	

			\begin{frame}[label=sectionIIsubsectionI]
				\frametitle{\sectionIIsubsectionItitle}
				\bigskip

				
				\btVFill
			\end{frame}

		% section II subsection II
		\subsection{\sectionIIsubsectionIItitle}\label{sectionIIsubsectionII}

			\begin{frame}
				\frametitle{\sectionIIsubsectionIItitle}
				\bigskip

		
				\btVFill 
			\end{frame}	

			\begin{frame}
				\frametitle{\sectionIIsubsectionIItitle} \small
				\bigskip


				\btVFill
			\end{frame}		


		% section II subsection III
		\subsection{\sectionIIsubsectionIIItitle}\label{sectionIIsubsectionIII}

			\begin{frame}
				\frametitle{\sectionIIsubsectionIIItitle}
				\bigskip

				\btVFill 
			\end{frame}

			\begin{frame}
				\frametitle{\sectionIIsubsectionIIItitle}\small
				\bigskip


				\btVFill 
			\end{frame}

			\begin{frame}
				\frametitle{\sectionIIsubsectionIIItitle} \scriptsize
				\bigskip
				
			
				\btVFill 
			\end{frame}

			\begin{frame}
				\frametitle{\sectionIIsubsectionIIItitle}
				\bigskip

				\btVFill 
			\end{frame}

		% section II subsection IV 
		\subsection{\sectionIIsubsectionIVtitle}\label{sectionIIsubsectionIV}

			\begin{frame}
				\frametitle{\sectionIIsubsectionIVtitle}
				\bigskip


				\btVFill 
			\end{frame}

			\begin{frame}
				\frametitle{\sectionIIsubsectionIVtitle}
				\bigskip


				\btVFill 
			\end{frame}

			\begin{frame}
				\frametitle{\sectionIIsubsectionIVtitle}
				\bigskip

				\btVFill 
			\end{frame}

		% section II subsection V 
		\subsection{\sectionIIsubsectionVtitle}\label{sectionIIsubsectionV}

			\begin{frame}
				\frametitle{\sectionIIsubsectionVtitle}
				\bigskip

			
				\btVFill 
			\end{frame}

			\begin{frame}
				\frametitle{\sectionIIsubsectionVtitle}
				\bigskip

				\btVFill 
			\end{frame}

			\begin{frame}
				\frametitle{\sectionIIsubsectionVtitle}\small
				\bigskip

				\btVFill 
			\end{frame}	
		
	% Section III
	\section{\sectionIIItitle}\label{sectionIII}

		% section III Outline
		\begin{frame}
			\large \textbf{Topic 3 - \sectionIIItitle} \vspace{3mm}\\

			\begin{itemize}
				\item \hyperlink{sectionIIIsubsectionI}{\sectionIIIsubsectionItitle} \vspc %  section III subsection I
				\item \hyperlink{sectionIIIsubsectionII}{\sectionIIIsubsectionIItitle} \vspc % section III subsection II
				\item \hyperlink{sectionIIIsubsectionIII}{\sectionIIIsubsectionIIItitle} \vspc % section III subsection III
				%\item \hyperlink{sectionIIIsubsectionIV}{\sectionIIIsubsectionIVtitle} \vspc % section III subsection IV
			\end{itemize}

		\end{frame}

		% section III subsection I
		\subsection{\sectionIIIsubsectionItitle}\label{sectionIIIsubsectionI}

			\begin{frame}
				\frametitle{\sectionIIIsubsectionItitle}
				\bigskip

					
				\btVFill
			\end{frame}

			\begin{frame}
				\frametitle{\sectionIIIsubsectionItitle}
				\bigskip

				\btVFill
			\end{frame}

		% section III subsection II
		\subsection{\sectionIIIsubsectionIItitle}\label{sectionIIIsubsectionII}	

			\begin{frame}
				\frametitle{\sectionIIIsubsectionIItitle}
				\bigskip


				\btVFill
			\end{frame}

			\begin{frame}
				\frametitle{\sectionIIIsubsectionIItitle}
				\bigskip
				

				\btVFill
			\end{frame}

			\begin{frame}
				\frametitle{\sectionIIIsubsectionIItitle}
				\bigskip
					
				
				\btVFill
			\end{frame}

		% section III subsection III
		\subsection{\sectionIIIsubsectionIIItitle}\label{sectionIIIsubsectionIII}

			\begin{frame}
				\frametitle{\sectionIIIsubsectionIIItitle}
				\bigskip

				
				\btVFill
			\end{frame}

			\begin{frame}
				\frametitle{\sectionIIIsubsectionIIItitle}
				\bigskip

 				
				\btVFill
			\end{frame}

			\begin{frame}
				\frametitle{\sectionIIIsubsectionIIItitle}
				\bigskip

				\btVFill
			\end{frame}



	% Section IV
	\section{\sectionIVtitle}\label{sectionIV}

		% section IV Outline
		\begin{frame}
			\large \textbf{Topic 3 - \sectionIVtitle} \vspace{3mm}\\

			\begin{itemize}
				\item \hyperlink{sectionIVsubsectionI}{\sectionIVsubsectionItitle} \vspc %  section IV subsection I
				\item \hyperlink{sectionIVsubsectionII}{\sectionIVsubsectionIItitle} \vspc % section IV subsection II
				\item \hyperlink{sectionIVsubsectionIII}{\sectionIVsubsectionIIItitle} \vspc % section IV subsection III
				\item \hyperlink{sectionIVsubsectionIV}{\sectionIVsubsectionIVtitle} \vspc % section IV subsection IV
			\end{itemize}

		\end{frame}

		% section IV subsection I
		\subsection{\sectionIVsubsectionItitle}\label{sectionIVsubsectionI}

			\begin{frame}
				\frametitle{\sectionIVsubsectionItitle}
				\bigskip

				\btVFill
			\end{frame}

			\begin{frame}
				\frametitle{\sectionIVsubsectionItitle}
				\bigskip


				\btVFill
			\end{frame}

		% section IV subsection II
		\subsection{\sectionIVsubsectionIItitle}\label{sectionIVsubsectionII}

			\begin{frame}
				\frametitle{\sectionIVsubsectionItitle}
				\bigskip


				\btVFill
			\end{frame}

			\begin{frame}
				\frametitle{\sectionIVsubsectionItitle}
				\bigskip


				\btVFill
			\end{frame}	


\end{document}





