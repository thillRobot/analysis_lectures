% Lecture Template for ME3001 - Mechanical Engineering Analysis - Tennessee Technological University
% Spring 2024 - condensing and streamlining lectures by combining topics into a single PDF under the module name
% this will simplify file and link management as well as make lectures easier to use in class
% - added image/ to clean directory and reduce redundancy, specific to module for now  
% Fall 2024 - typeset Curve Fitting notes for the first timei

% Module Name: - Lecture Module Template
% Topic 1 - 
% Topic 2 - 
% Topic 3 - 
% Topic 4 - 

\documentclass[fleqn]{beamer} % for presentation (has nav buttons at bottom)

\usepackage{../analysis_lectures} % sty in the parent directory

\author{ME3001 - Mechanical Engineering Analysis}

\newcommand{\MNUM}{5\hspace{2mm}} % module number 
\newcommand{\moduletitle}{Numerical Integration and Curve Fitting}

\newcommand{\sectionItitle}{Overview and Motivation}
\newcommand{\sectionIItitle}{Linear Regression}
\newcommand{\sectionIIItitle}{Polynomial Splines}
\newcommand{\sectionIVtitle}{Lagrange Polynomials}

\newcommand{\sectionIsubsectionItitle}{Problem Definition}
\newcommand{\sectionIsubsectionIItitle}{Engineering Applications}
\newcommand{\sectionIsubsectionIIItitle}{}
\newcommand{\sectionIsubsectionIVtitle}{}

\newcommand{\sectionIIsubsectionItitle}{Overview}
\newcommand{\sectionIIsubsectionIItitle}{Fit Criteria}
\newcommand{\sectionIIsubsectionIIItitle}{Linear Least Squares}
\newcommand{\sectionIIsubsectionIVtitle}{}

\newcommand{\sectionIIIsubsectionItitle}{}
\newcommand{\sectionIIIsubsectionIItitle}{}
\newcommand{\sectionIIIsubsectionIIItitle}{}
\newcommand{\sectionIIIsubsectionIVtitle}{}

\newcommand{\sectionIVsubsectionItitle}{}
\newcommand{\sectionIVsubsectionIItitle}{}
\newcommand{\sectionIVsubsectionIIItitle}{}
\newcommand{\sectionIVsubsectionIVtitle}{}

% custom box
\newsavebox{\mybox}

\title{Lecture Module - \moduletitle}

\date{Mechanical Engineering\vspc Tennessee Technological University}

\begin{document}

	\lstset{language=MATLAB,basicstyle=\ttfamily\small,showstringspaces=false}

	\frame{\titlepage \center\begin{framed}\Large \textbf{Module \MNUM - \moduletitle}\end{framed} \vspace{5mm}}

	% Module Outline
	\begin{frame} 
		\large \textbf{Module \MNUM - \moduletitle} \vspace{3mm}\\

		\begin{itemize}
			\item Topic 1 - \hyperlink{sectionI}{\sectionItitle} \vspc % section I
			\item Topic 2 - \hyperlink{sectionII}{\sectionIItitle} \vspc % section II
			\item Topic 3 - \hyperlink{sectionIII}{\sectionIIItitle} \vspc % section III
			\item Topic 4 - \hyperlink{sectionIV}{\sectionIIItitle} \vspc % section IV
		\end{itemize}

	\end{frame}

	% section I
	\section{\sectionItitle}\label{sectionI}

		% section I Outline
		\begin{frame} 
			\large \textbf{Topic 1 - \sectionItitle} \vspace{3mm}\\

			\begin{itemize}
				\item \hyperlink{sectionIsubsectionI}{\sectionIsubsectionItitle} \vspc %  section I subsection I
				\item \hyperlink{sectionIsubsectionII}{\sectionIsubsectionIItitle} \vspc % section I subsection II
				\item \hyperlink{sectionIsubsectionIII}{\sectionIsubsectionIIItitle} \vspc % section I subsection III
				\item \hyperlink{sectionIsubsectionIV}{\sectionIsubsectionIVtitle} \vspc % section I subsection IV
			\end{itemize}
		\end{frame}
	

		% section I subsection I 
		\subsection{\sectionIsubsectionItitle}\label{sectionIsubsectionI}

			\begin{frame}
				\frametitle{\sectionIsubsectionItitle}
				\bigskip
  
        \textbf{What is curve fitting}?
        
        - various techniques to fit a curve or function to discrete data \vspace{5mm}
 
        - "Data is often given for discrete values along a continuum. However, you may require
estimates at points between the discrete values" - \underline{ Numerical Methods for Engineers}, Chapra and Canale \vspace{5mm}\\ 			
        - additional problem is to find a simpler form of a complicated function by fitting function to data sampled from original function
	
				\btVFill
			\end{frame}

			\begin{frame}
				\frametitle{\sectionIsubsectionItitle}
				\bigskip

        \textbf{Two General Approaches}
        \begin{itemize}
          \item 1) Given data with random error, find a single curve that represents the overall trend of the data. \vspace{2mm}\\ 
          -  "Because any individual data point may be incorrect, we make no effort to intersect every point" - \underline{ Numerical Methods for Engineers}, Chapra and Canale \vspc 
        \item 2) Given data assumed to be precise or specified, find a curve that directly passes through each data point \vspace{5mm}\\ 
			  \end{itemize}	
				\btVFill
			\end{frame}


		% section I subsection II
		\subsection{\sectionIsubsectionIItitle}\label{sectionIsubsectionII}

			\begin{frame}
				\frametitle{\sectionIsubsectionIItitle} \small
				\bigskip
        \textbf{Example Applications in Engineering}
        \begin{itemize}
          
          \item Calibration Curves, Sensors and Instrumentation
          \item Table Interpolation, Mechanics, Thermo, Statistics
          \item Velocity Profile Generation, Dynamics of Machinery, Robotics
        
        \end{itemize}

        \textbf{Two General Problems}        
        \begin{itemize}
          \item Trend Analysis - predictions from dataset using interpolation polynomial or lsr
          \item Hypothesis Testing - compare predicted to measured data for model performance or selection
        \end{itemize}
				
				\btVFill
			\end{frame}


		% section I subsection III
		\subsection{\sectionIsubsectionIIItitle}\label{sectionIsubsectionIII}
			\begin{frame} 
				\frametitle{\sectionIsubsectionIIItitle}
				\bigskip

				
				\btVFill
			\end{frame}	

			\begin{frame} 
				\frametitle{\sectionIsubsectionIIItitle}
				\bigskip

			 
				\btVFill
			\end{frame}	


		% section I subsection IV
		\subsection{\sectionIsubsectionIVtitle}\label{sectionIsubsectionIV}	

			\begin{frame}
				\frametitle{\sectionIsubsectionIVtitle}
				\bigskip


				\btVFill
			\end{frame}
	
			\begin{frame}
				\frametitle{\sectionIsubsectionIVtitle}
				\bigskip


				\btVFill
			\end{frame}


	% Section II
	\section{\sectionIItitle}\label{sectionII}

		% section II Outline
		\begin{frame}
			\large \textbf{Topic 2 - \sectionIItitle} \vspace{3mm}\\

			\begin{itemize}
				\item \hyperlink{sectionIIsubsectionI}{\sectionIIsubsectionItitle} \vspc %  section II subsection I
				\item \hyperlink{sectionIIsubsectionII}{\sectionIIsubsectionIItitle} \vspc % section II subsection II
				\item \hyperlink{sectionIIsubsectionIII}{\sectionIIsubsectionIIItitle} \vspc % section II subsection III
				\item \hyperlink{sectionIIsubsectionIV}{\sectionIIsubsectionIVtitle} \vspc % section II 
			\end{itemize}

		\end{frame}

		% section II subsection I
		\subsection{\sectionIIsubsectionItitle}\label{sectionIIsubsectionI}

			\begin{frame}[label=sectionIIsubsectionI]
				\frametitle{\sectionIIsubsectionItitle}
				\bigskip
        Consider fitting a straight line to a dataset
        \[(x_1, y_1), (x_2, y_2), ..., (x_n,y_n)\] 
        with a function \[y=a_o+a_1+e\]

        This can be rearranged to show the {\bf error} as

        \[e=y-a_0-a_1x\]
  
        The general problem is to find a function that minimizes the error  
  
				\btVFill
			\end{frame}

			\begin{frame}[label=sectionIIsubsectionI]
				\frametitle{\sectionIIsubsectionItitle}
				\bigskip

        To find the coefficents of the fit line, the minimization objective must be considered carefully. You might consider fitting a model that mimizes the error directly, but this will not work. The absolute value approach is also problematic.

          \begin{itemize}
            \item \[\Sigma_{i=1}^n e_i = \left(y_i-a_0-a_1x_i\right)\] \vspace{3mm}
            \item \[\Sigma_{i=1}^n |e_i| = \left|y_i-a_0-a_1x_i\right|\] \vspace{3mm}
          \end{itemize}
      
 To solve these issues, the common technique is to \underline{\hspace{25mm}} the error.
          \begin{itemize}
             \item \[\Sigma_{i=1}^n e_i^2 = \left(y_i-a_0-a_1x_i\right)^2\] \vspace{3mm}
          \end{itemize}

				\btVFill
			\end{frame}	

			\begin{frame}[label=sectionIIsubsectionI]
				\frametitle{\sectionIIsubsectionItitle}
				\bigskip

				
				\btVFill
			\end{frame}

		% section II subsection II
		\subsection{\sectionIIsubsectionIItitle}\label{sectionIIsubsectionII}

			\begin{frame}
				\frametitle{\sectionIIsubsectionIItitle} \small
				\bigskip

        To fit a straight line to the data, we must find the values $a_o$ and $a_1$ that minimize the square of the error. First find the partial derivatives of the sqaured error and set these equal to zero

\[ S_r=\Sigma_{i=1}^ne_i^2=\left(y_i-a_0-a_1x_i\right)^2 \]
    
\[ \frac{\delta S_r}{\delta a_0}  \]

			\btVFill 
			\end{frame}	

			\begin{frame}
				\frametitle{\sectionIIsubsectionIItitle} \small
				\bigskip


				\btVFill
			\end{frame}		


		% section II subsection III
		\subsection{\sectionIIsubsectionIIItitle}\label{sectionIIsubsectionIII}

			\begin{frame}
				\frametitle{\sectionIIsubsectionIIItitle} \small
				\bigskip

				
				\btVFill 
			\end{frame}

			\begin{frame}
				\frametitle{\sectionIIsubsectionIIItitle}\small
				\bigskip


				\btVFill 
			\end{frame}


		% section II subsection IV 
		\subsection{\sectionIIsubsectionIVtitle}\label{sectionIIsubsectionIV}

			\begin{frame}
				\frametitle{\sectionIIsubsectionIVtitle}
				\bigskip

				
				\btVFill 
			\end{frame}

			\begin{frame}
				\frametitle{\sectionIIsubsectionIVtitle}
				\bigskip


				\btVFill 
			\end{frame}
		

	% Section III
	\section{\sectionIIItitle}\label{sectionIII}

		% section III Outline
		\begin{frame}
			\large \textbf{Topic 3 - \sectionIIItitle} \vspace{3mm}\\

			\begin{itemize}
				\item \hyperlink{sectionIIIsubsectionI}{\sectionIIIsubsectionItitle} \vspc %  section III subsection I
				\item \hyperlink{sectionIIIsubsectionII}{\sectionIIIsubsectionIItitle} \vspc % section III subsection II
				\item \hyperlink{sectionIIIsubsectionIII}{\sectionIIIsubsectionIIItitle} \vspc % section III subsection III
				\item \hyperlink{sectionIIIsubsectionIV}{\sectionIIIsubsectionIVtitle} \vspc % section III subsection IV
			\end{itemize}

		\end{frame}


		% section III subsection I
		\subsection{\sectionIIIsubsectionItitle}\label{sectionIIIsubsectionI}

			\begin{frame}
				\frametitle{\sectionIIIsubsectionItitle}
				\bigskip

			  	
				\btVFill
			\end{frame}

			\begin{frame}
				\frametitle{\sectionIIIsubsectionItitle}
				\bigskip

			  
				\btVFill
			\end{frame}


		% section III subsection II
		\subsection{\sectionIIIsubsectionIItitle}\label{sectionIIIsubsectionII}	

			\begin{frame}
				\frametitle{\sectionIIIsubsectionIItitle}
				\bigskip

				
				\btVFill
			\end{frame}

			\begin{frame}
				\frametitle{\sectionIIIsubsectionIItitle}
				\bigskip
				

				\btVFill
			\end{frame}


		% section III subsection III
		\subsection{\sectionIIIsubsectionIIItitle}\label{sectionIIIsubsectionIII}

			\begin{frame}
				\frametitle{\sectionIIIsubsectionIIItitle}
				\bigskip


				\btVFill
			\end{frame}

			\begin{frame}
				\frametitle{\sectionIIIsubsectionIIItitle}
				\bigskip


				\btVFill
			\end{frame}


		% section III subsection IV
		\subsection{\sectionIIIsubsectionIVtitle}\label{sectionIIIsubsectionIV}

			\begin{frame}
				\frametitle{\sectionIIIsubsectionIVtitle}
				\bigskip


				\btVFill
			\end{frame}	

			\begin{frame}
				\frametitle{\sectionIIIsubsectionIVtitle} \small
				\bigskip


				\btVFill
			\end{frame}	

	
	% Section IV
	\section{\sectionIVtitle}\label{sectionIV}

		% section IV Outline
		\begin{frame}
			\large \textbf{Topic 3 - \sectionIVtitle} \vspace{3mm}\\

			\begin{itemize}
				\item \hyperlink{sectionIVsubsectionI}{\sectionIVsubsectionItitle} \vspc %  section IV subsection I
				\item \hyperlink{sectionIVsubsectionII}{\sectionIVsubsectionIItitle} \vspc % section IV subsection II
				\item \hyperlink{sectionIVsubsectionIII}{\sectionIVsubsectionIIItitle} \vspc % section IV subsection III
				\item \hyperlink{sectionIVsubsectionIV}{\sectionIVsubsectionIVtitle} \vspc % section IV subsection IV
			\end{itemize}

		\end{frame}

		% section IV subsection I
		\subsection{\sectionIVsubsectionItitle}\label{sectionIVsubsectionI}

			\begin{frame}
				\frametitle{\sectionIVsubsectionItitle}
				\bigskip

		
				\btVFill
			\end{frame}

		% section IV subsection II
		\subsection{\sectionIVsubsectionIItitle}\label{sectionIVsubsectionII}

			\begin{frame}
				\frametitle{\sectionIVsubsectionIItitle}
				\bigskip


				\btVFill
			\end{frame}

			\begin{frame}
				\frametitle{\sectionIVsubsectionIItitle}
				\bigskip


				\btVFill
			\end{frame}	


		% section IV subsection III
		\subsection{\sectionIVsubsectionIIItitle}\label{sectionIVsubsectionIII}

			\begin{frame}
				\frametitle{\sectionIVsubsectionIIItitle}
				\bigskip

				
				\btVFill
			\end{frame}	

			\begin{frame}
				\frametitle{\sectionIVsubsectionIIItitle}
				\bigskip

				
				\btVFill
			\end{frame}	

\end{document}





