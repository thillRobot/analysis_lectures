% Lecture Template for ME3001 - Mechanical Engineering Analysis - Tennessee Technological University
% Spring 2024 - condensing and streamlining lectures by combining topics into a single PDF under the module name
% this will simplify file and link management as well as make lectures easier to use in class
% - added image/ to clean directory and reduce redundancy, specific to module for now  
% Fall 2024 - typeset Curve Fitting notes for the first time

% Module Name: - Lecture Module Template
% Topic 1 - Overview and Motivation
% Topic 2 - Linear Regression
% Topic 3 - Polynomial Splines
% Topic 4 - Lagrange Polynomial

\documentclass[fleqn]{beamer} % for presentation (has nav buttons at bottom)

\usepackage{../analysis_lectures} % sty in the parent directory

\author{ME3001 - Mechanical Engineering Analysis}

\newcommand{\MNUM}{5\hspace{2mm}} % module number 
\newcommand{\moduletitle}{Numerical Integration and Curve Fitting}

\newcommand{\sectionItitle}{Overview and Motivation}
\newcommand{\sectionIItitle}{Linear Regression}
\newcommand{\sectionIIItitle}{Interpolation and Splines}
\newcommand{\sectionIVtitle}{Lagrange Polynomials}

\newcommand{\sectionIsubsectionItitle}{Problem Definition}
\newcommand{\sectionIsubsectionIItitle}{Engineering Applications}
\newcommand{\sectionIsubsectionIIItitle}{}
\newcommand{\sectionIsubsectionIVtitle}{}

\newcommand{\sectionIIsubsectionItitle}{Overview}
\newcommand{\sectionIIsubsectionIItitle}{Fit Criteria}
\newcommand{\sectionIIsubsectionIIItitle}{Linear Least Squares}
\newcommand{\sectionIIsubsectionIVtitle}{MATLAB Example}

\newcommand{\sectionIIIsubsectionItitle}{Polynomial Interpolation Functions}
\newcommand{\sectionIIIsubsectionIItitle}{Polynomial Splines}
\newcommand{\sectionIIIsubsectionIIItitle}{Linear Splines}
\newcommand{\sectionIIIsubsectionIVtitle}{Cubic Splines}

\newcommand{\sectionIVsubsectionItitle}{}
\newcommand{\sectionIVsubsectionIItitle}{}
\newcommand{\sectionIVsubsectionIIItitle}{}
\newcommand{\sectionIVsubsectionIVtitle}{}

% custom box
\newsavebox{\mybox}

\title{Lecture Module - \moduletitle}

\date{Mechanical Engineering\vspc Tennessee Technological University}

\begin{document}
  
  \footnotesize
	%\lstset{language=MATLAB,basicstyle=\ttfamily\small,showstringspaces=false}

	\frame{\titlepage \center\begin{framed}\Large \textbf{Module \MNUM - \moduletitle}\end{framed} \vspace{5mm}}

	% Module Outline
	\begin{frame} 
		\large \textbf{Module \MNUM - \moduletitle} \vspace{3mm}\\

		\begin{itemize}
			\item Topic 1 - \hyperlink{sectionI}{\sectionItitle} \vspc % section I
			\item Topic 2 - \hyperlink{sectionII}{\sectionIItitle} \vspc % section II
			\item Topic 3 - \hyperlink{sectionIII}{\sectionIIItitle} \vspc % section III
			\item Topic 4 - \hyperlink{sectionIV}{\sectionIVtitle} \vspc % section IV
		\end{itemize}

	\end{frame}

	% section I
	\section{\sectionItitle}\label{sectionI}

		% section I Outline
		\begin{frame} 
			\large \textbf{Topic 1 - \sectionItitle} \vspace{3mm}\\

			\begin{itemize}
				\item \hyperlink{sectionIsubsectionI}{\sectionIsubsectionItitle} \vspc %  section I subsection I
				\item \hyperlink{sectionIsubsectionII}{\sectionIsubsectionIItitle} \vspc % section I subsection II
				\item \hyperlink{sectionIsubsectionIII}{\sectionIsubsectionIIItitle} \vspc % section I subsection III
				\item \hyperlink{sectionIsubsectionIV}{\sectionIsubsectionIVtitle} \vspc % section I subsection IV
			\end{itemize}
		\end{frame}
	

		% section I subsection I 
		\subsection{\sectionIsubsectionItitle}\label{sectionIsubsectionI}

			\begin{frame}
				\frametitle{\sectionIsubsectionItitle}
				\bigskip
  
        \textbf{\small What is curve fitting}?
        \vspace{2mm}\\
        
        - various techniques to fit a curve or function to discrete data \vspace{5mm}
 
        - "Data is often given for discrete values along a continuum. However, you may require
estimates at points between the discrete values" - \underline{ Numerical Methods for Engineers}, Chapra and Canale \vspace{5mm}\\ 			
        - additional problem is to find a simpler form of a complicated function by fitting function to data sampled from original function
	
				\btVFill
			\end{frame}

			\begin{frame}
				\frametitle{\sectionIsubsectionItitle} \footnotesize
				\bigskip

        \textbf{\small Two General Approaches}
        \vspace{2mm}
        \begin{itemize}
          \item 1) Given data with random error, find a single curve that represents the overall trend of the data. \vspace{2mm}\\ 
          -  "Because any individual data point may be incorrect, we make no effort to intersect every point" - \underline{ Numerical Methods for Engineers}, Chapra and Canale \vspace{1mm}\\
          - Common method is {\it regression} (LSR) \vspace{3mm}\\
        \item 2) Given data assumed to be precise or specified, find a curve that directly passes through each data point \vspace{1mm}\\
          - Known as {\it interpolation}, {\it extrapolation} 
			  \end{itemize}	
				\btVFill
			\end{frame}


		% section I subsection II
		\subsection{\sectionIsubsectionIItitle}\label{sectionIsubsectionII}

			\begin{frame}
				\frametitle{\sectionIsubsectionIItitle}
				\bigskip
        \textbf{\small Example Applications in Engineering}
        \begin{itemize}
          
          \item Calibration Curves, Sensors and Instrumentation
          \item Table Interpolation, Mechanics, Thermo, Statistics
          \item Velocity Profile Generation, Dynamics of Machinery, Robotics
        
        \end{itemize}
        \vspace{5mm}
        \textbf{\small Two General Problems}        
        \begin{itemize}
          \item Trend Analysis - predictions from dataset using interpolation polynomial or LSR
          \item Hypothesis Testing - compare predicted to measured data for model performance or selection
        \end{itemize}
				
				\btVFill
			\end{frame}


		% % section I subsection III
		% \subsection{\sectionIsubsectionIIItitle}\label{sectionIsubsectionIII}
			% \begin{frame} 
				% \frametitle{\sectionIsubsectionIIItitle}
				% \bigskip

				
				% \btVFill
			% \end{frame}	

			% \begin{frame} 
				% \frametitle{\sectionIsubsectionIIItitle}
				% \bigskip

			 
				% \btVFill
			% \end{frame}	


		% % section I subsection IV
		% \subsection{\sectionIsubsectionIVtitle}\label{sectionIsubsectionIV}	

			% \begin{frame}
				% \frametitle{\sectionIsubsectionIVtitle}
				% \bigskip


				% \btVFill
			% \end{frame}
	
			% \begin{frame}
				% \frametitle{\sectionIsubsectionIVtitle}
				% \bigskip


				% \btVFill
			% \end{frame}


	% Section II
	\section{\sectionIItitle}\label{sectionII}

		% section II Outline
		\begin{frame}
			\large \textbf{Topic 2 - \sectionIItitle} \vspace{3mm}\\

			\begin{itemize}
				\item \hyperlink{sectionIIsubsectionI}{\sectionIIsubsectionItitle} \vspc %  section II subsection I
				\item \hyperlink{sectionIIsubsectionII}{\sectionIIsubsectionIItitle} \vspc % section II subsection II
				\item \hyperlink{sectionIIsubsectionIII}{\sectionIIsubsectionIIItitle} \vspc % section II subsection III
				\item \hyperlink{sectionIIsubsectionIV}{\sectionIIsubsectionIVtitle} \vspc % section II 
			\end{itemize}

		\end{frame}

		% section II subsection I
		\subsection{\sectionIIsubsectionItitle}\label{sectionIIsubsectionI}

			\begin{frame}[label=sectionIIsubsectionI]
				\frametitle{\sectionIIsubsectionItitle}
				\bigskip
        Consider fitting a straight line to a dataset
        \[(x_1, y_1), (x_2, y_2), ..., (x_n,y_n)\] 
        with a function \[y=a_o+a_1\]

        A value $y$ can be defined in terms of the function with an error term $e$

        \[y=a_0+a_1+e\]

        This can be rearranged to show the {\bf error} as

        \[e=y-a_0-a_1x\]
  
        The goal is to find the coefficients of a function that minimizes the error  
  
				\btVFill
			\end{frame}

			% \begin{frame}[label=sectionIIsubsectionI]
				% \frametitle{\sectionIIsubsectionItitle}
				% \bigskip

        % To find the coefficents of the fit line, the minimization objective must be considered carefully. You might consider fitting a model that mimizes the error directly, but this will not work. The absolute value approach is also problematic.

          % \begin{itemize}
            % \item \[\Sigma_{i=1}^n e_i = \left(y_i-a_0-a_1x_i\right)\] \vspace{3mm}
            % \item \[\Sigma_{i=1}^n |e_i| = \left|y_i-a_0-a_1x_i\right|\] \vspace{3mm}
          % \end{itemize}
      
 % To solve these issues, the common technique is to \underline{\hspace{25mm}} the error.
          % \begin{itemize}
             % \item \[\Sigma_{i=1}^n e_i^2 = \left(y_i-a_0-a_1x_i\right)^2\] \vspace{3mm}
          % \end{itemize}

				% \btVFill
			% \end{frame}	

			% \begin{frame}[label=sectionIIsubsectionI]
				% \frametitle{\sectionIIsubsectionItitle}
				% \bigskip

				
				% \btVFill
			% \end{frame}

		%section II subsection II
		\subsection{\sectionIIsubsectionIItitle}\label{sectionIIsubsectionII}

			\begin{frame}
				\frametitle{\sectionIIsubsectionIItitle}
				\bigskip
				
				To find the coefficents of the fit line, the minimization objective must be considered carefully. You might consider fitting a model that mimizes the error directly, but this will not work. The absolute value approach is also problematic.
\vspace{2mm}
          \begin{itemize}
            \item \[\Sigma_{i=1}^n e_i = \left(y_i-a_0-a_1x_i\right)\] \vspace{3mm}
            \item \[\Sigma_{i=1}^n |e_i| = \left|y_i-a_0-a_1x_i\right|\] \vspace{3mm}
          \end{itemize}
 \vspace{3mm}    
 To solve these issues, the common technique is to \underline{\hspace{25mm}} the error.\vspace{3mm}
          \begin{itemize}
             \item \[\Sigma_{i=1}^n e_i^2 = \left(y_i-a_0-a_1x_i\right)^2\] \vspace{3mm}
          \end{itemize}

       
			\btVFill 
			\end{frame}	

		% section II subsection III
		\subsection{\sectionIIsubsectionIIItitle}\label{sectionIIsubsectionIII}

			\begin{frame}
				\frametitle{\sectionIIsubsectionIIItitle}
				\bigskip
				
				To fit a straight line to the data, we must find the values $a_o$ and $a_1$ that minimize the square of the error. First find the partial derivatives of the sqaured error and set these equal to zero

				\[ S_r=\Sigma_{i=1}^ne_i^2=\left(y_i-a_0-a_1x_i\right)^2 \]
					
				\[ \frac{\delta S_r}{\delta a_0} = -2 \Sigma\left( y_i - a_0 - a_1x_i \right) \]
				\[ \frac{\delta S_r}{\delta a_1} = -2 \Sigma\left[ \left( y_i - a_0 - a_1x_i\right)x_i\right]\]

				\[ 0 = \Sigma y_i - \Sigma a_0 - \Sigma a_1x_i \]
				\[ 0 = \Sigma y_ix_i - \Sigma a_0x_i - \Sigma a_ix_i^2\]

				
				\btVFill 
			\end{frame}

			\begin{frame}
				\frametitle{\sectionIIsubsectionIIItitle}
				\bigskip
	
				Use $\Sigma a_0=na_0$ and the resulting equations can be solved as a linear system in terms of the coefficients $a_0$, $a_1$, and number of data points $n$. 
				\[ 0 = \Sigma y_i - \Sigma a_0 - \Sigma a_1x_i \]
				\[ 0 = \Sigma y_ix_i - \Sigma a_0x_i - \Sigma a_ix_i^2\]



				This leads to the standard equations 
				\[a_1= \frac{n\Sigma x_iy_i-\Sigma x_i \Sigma y_i}{n\Sigma x_i^2-\left(\Sigma x_i^2\right)} \]
				\[a_0= \bar{y} -a_1\bar{x}\]
				
				This alternate form can be found by multipying by $1=\frac{-1}{-1}$
				\[a_1=\frac{\Sigma x_i\Sigma y_i-n\Sigma x_i y_i }{(\Sigma x_i)^2-n\Sigma x_i^2}\]
				\[a_0=\frac{\Sigma x_i\Sigma x_i y_i-\Sigma x_i^2\Sigma y_i}{(\Sigma x_i)^2-n\Sigma x_i^2}\] 

				\btVFill 
			\end{frame}


		% section II subsection IV 
		\subsection{\sectionIIsubsectionIVtitle}\label{sectionIIsubsectionIV}

			\begin{frame}[fragile]
				\frametitle{\sectionIIsubsectionIVtitle}
				\bigskip
				This standard technique is built into the MATLAB function {\it polyfit}. This function can also be used for higher order regression lines.
				\begin{lstlisting}
% ME3001, TNTech, Tristan Hill, October 29, 2024
% Curve fitting with Linear Regression 
% this program will
% 1) generate dataset with random noise
% 2) find best fit using 'linear least sqaures regression' from eqs in notes
% 3) find best fit using LSR with MATLAB polyfit()
clear; clc; close all  
        \end{lstlisting}
          
				\btVFill 
			\end{frame}

			\begin{frame}[fragile]
				\frametitle{\sectionIIsubsectionIVtitle}
				\bigskip

        \begin{lstlisting}
% step 1) - generate dataset
m=-3; b=1.5;
error_scale=5;

xdata=-5:.5:5;
n=length(xdata);
ydata=m*xdata+b+rand(1,n)*error_scale;

figure(1); hold on
plot(xdata,ydata,'o')
grid on 
        \end{lstlisting}

				\btVFill 
			\end{frame}
		

			\begin{frame}[fragile]
				\frametitle{\sectionIIsubsectionIVtitle}
				\bigskip

        \begin{lstlisting}
% step 2) - fit line with LSR equations
a1=(n*sum(xdata.*ydata)-sum(xdata)*sum(ydata))/...
    (n*sum(xdata.^2)-sum(xdata.^2))
a0=sum(ydata)/n

% compare with equations from ME3023
a1=(sum(xdata)*sum(ydata)-n*sum(xdata.*ydata))/...
    (sum(xdata)^2-n*sum(xdata.^2))
a0=(sum(xdata)*sum(xdata.*ydata)-sum(xdata.^2)*sum(ydata))/...
    (sum(xdata)^2-n*sum(xdata.^2))          
        \end{lstlisting}

				\btVFill 
			\end{frame}

			\begin{frame}[fragile]
				\frametitle{\sectionIIsubsectionIVtitle}
				\bigskip

        \begin{lstlisting}
% compute and plot values on the best fit line
xfit=-5:.1:5;
yfit=a1*xfit+a0;

plot(xfit,yfit,'-')

% step 3) - fit line with LSR in MATLAB
A=polyfit(xdata,ydata,1) % get second the coefficients

pfit=A(2)+A(1)*xfit;  % calculate points on curve
plot(xfit,pfit,':g','LineWidth',5)
        \end{lstlisting}

        Download {\it linear\_regression\_example1.m} for the complete program.
				\btVFill 
			\end{frame}

	% Section III
	\section{\sectionIIItitle}\label{sectionIII}

		% section III Outline
		\begin{frame}
			\large \textbf{Topic 3 - \sectionIIItitle} \vspace{3mm}\\

			\begin{itemize}
				\item \hyperlink{sectionIIIsubsectionI}{\sectionIIIsubsectionItitle} \vspc %  section III subsection I
				\item \hyperlink{sectionIIIsubsectionII}{\sectionIIIsubsectionIItitle} \vspc % section III subsection II
				\item \hyperlink{sectionIIIsubsectionIII}{\sectionIIIsubsectionIIItitle} \vspc % section III subsection III
				\item \hyperlink{sectionIIIsubsectionIV}{\sectionIIIsubsectionIVtitle} \vspc % section III subsection IV
			\end{itemize}

		\end{frame}


		% section III subsection I
		\subsection{\sectionIIIsubsectionItitle}\label{sectionIIIsubsectionI}

			\begin{frame}
				\frametitle{\sectionIIIsubsectionItitle}
				\bigskip

			  	
				\btVFill
			\end{frame}

			\begin{frame}
				\frametitle{\sectionIIIsubsectionItitle}
				\bigskip

			  
				\btVFill
			\end{frame}


		% section III subsection II
		\subsection{\sectionIIIsubsectionIItitle}\label{sectionIIIsubsectionII}	

			\begin{frame}
				\frametitle{\sectionIIIsubsectionIItitle}
				\bigskip

				
				\btVFill
			\end{frame}

			\begin{frame}
				\frametitle{\sectionIIIsubsectionIItitle}
				\bigskip
				

				\btVFill
			\end{frame}


		% section III subsection III
		\subsection{\sectionIIIsubsectionIIItitle}\label{sectionIIIsubsectionIII}

			\begin{frame}
				\frametitle{\sectionIIIsubsectionIIItitle}
				\bigskip
        
        Fit a {\bf spline} function consisting of multiple {\it linear} functions to a set of data. \\
        
        \begin{itemize}
          \item the spline must pass through $n$ data points $\left(x_i,f_i \right)_{i=1,2,...,n}$
          \item $n-1$ intervals are defined by spline functions $s_i(x)_{i=1,2,...,n-1}$
          \[s_i\left(x\right)=a_i+b_i(x-x_i) \]
          \[a_i=f_i\] \[b_i=\frac{f_{i+1}-f_i}{x_{i+1}-x_i}\]
          \item the $(x-x_i)$ term handles the shift to the $i^{th}$ spline function
          \item substitute $b_i$ into $s_i(x)$ to get the following description of the spline
          \[s_i\left(x\right)=f_i+\left(\frac{f_{i+1}-f_i}{x_{i+1}-x_i}\right)\left(x-x_i\right) \]
        \end{itemize}

				\btVFill
        \tiny{Derivation: \underline{Applied Numerical Methods with MATLAB} Steven Chapra}
			
    \end{frame}

		\begin{frame}
				\frametitle{\sectionIIIsubsectionIIItitle}
				\bigskip
        
			
    \end{frame}
		% section III subsection IV
		\subsection{\sectionIIIsubsectionIVtitle}\label{sectionIIIsubsectionIV}

			\begin{frame}
				\frametitle{\sectionIIIsubsectionIVtitle}
				\bigskip
        
        "Cubic splines are most commonly used in practice" \vspcc
          
        Fit a {\bf spline} function consisting of multiple {\it cubic} functions to the data
       
        \begin{itemize}
          \item the spline must pass through $n$ data points $\left(x_i,f_i \right)_{i=1,2,...,n}$ 
          \item $n-1$ intervals are defined by spline functions $s_i(x)_{i=1,2,...,n-1}$
          \[s_i\left(x\right)=a_i+b_i\left(x-x_i\right)+c_i\left(x-x_i\right)^2+d_i\left(x-x_i\right)^3\]
          \item coeficients $a_i,b_i,c_i,d_i$ must be found $\implies 4\left(n-1\right)$ unknowns 
          \item the slope at each point must match for a smooth spline  
          \item two additional conditions are required due to no slope match at ends
          \[2*\left(n-1\right)+2*\left(n-1\right)-2+2=4\left(n-1\right)\]
       \end{itemize}            
 
				\btVFill
			\end{frame}	

			\begin{frame}
				\frametitle{\sectionIIIsubsectionIVtitle} 
				\bigskip

        The functions passes through all the data points
        \[s_i\left(x\right)=a_i+b_i\left(x-x_i\right)+c_i\left(x-x_i\right)^2+d_i\left(x-x_i\right)^3\]
        \[s_i\left(x_i\right)=f_i\]       
        \[f_i=a_i+b_i\left(x_i-x_i\right)+c_i\left(x_i-x_i\right)^2+d_i\left(x_i-x_i\right)^3=a_i\]
        The $a_i$ coeficients can be replaced with the function values $f_i$
        \[f_i=a_i\]
        \[s_i\left(x\right)=f_i+b_i\left(x-x_i\right)+c_i\left(x-x_i\right)^2+d_i\left(x-x_i\right)^3\]
        Define the $i^{th}$ stepsize for convenience
        \[h_i=x_{i+1}-x_i\]
        \btVFill
			\end{frame}	

			\begin{frame}
				\frametitle{\sectionIIIsubsectionIVtitle} 
				\bigskip
        
        The function values are equal at each point
        \[f_i+b_i\left(h_i\right)+c_i\left(h_i\right)^2+d_i\left(h_i\right)^3=f_{i+1}\]
      
        The slope (first derivative) matches at each point between intervals 
        \[s_i'\left(x\right)=b_i+2c_i\left(x-x_i\right)+3d_i\left(x-x_i\right)^2\]
        \[b_i+2c_i\left(x_{i+1}-x_i\right)+3d_i\left(x_{i+1}-x_i\right)^2\]
        \[= b_{i+1}+2c_{i+1}\left(x_{i+1}-x_{i+1}\right)+d_{i+1}\left(x_{i+1}-x_{i+1}\right)\] 
        \[\implies b_i+2c_ih_i+3d_ih_i^2=b_{i+1}\]
  
        The second derivative is also matches at the nodes for a {\it natural spline} 
        \[s_i''\left(x\right)=2c_i+6d_i\left(x-x_i\right)\]
        \[2c_i + 6d_ih_i=2c_{i+1} + 6d_i\left(x_{i+1}-x_{i+1}\right) \implies d_i=\frac{c_{i+1}-c_i}{3h_i}\]

        \btVFill
			\end{frame}	
	
			\begin{frame}
				\frametitle{\sectionIIIsubsectionIVtitle} 
				\bigskip
        Subsitute $d_i$ and solve for $b_i$ 
        \[f_i+b_ih_i+c_ih_i^2+\left(\frac{c_{i+1}-c_i}{3h_i}\right)h_i^3=f_{i+1}\]
        \[f_i+b_ih_i+\frac{h_i^2}{3}\left(2c_i+c_{i+1}\right)=f_{i+1}\]
        \[\implies b_i=\frac{f_{i+1}-f_i}{h_i}-\frac{h_i}{3}\left(2c_i+c_{i+1}\right)\]
        repeat for derivative condition equation
        \[b_i+2c_ih_i+3\left(\frac{c_{i+1}-c_i}{3h_i}\right)h_i^2=b_{i+1}\implies b_{i+1}=b_i+h_i\left(c_i+c_{i+1}\right)\]

				\btVFill
			\end{frame}	

			\begin{frame}
				\frametitle{\sectionIIIsubsectionIVtitle} 
				\bigskip
        Use the result from above
        \[b_i=\frac{f_{i+1}-f_i}{h_i}-\frac{h_i}{3}\left(2c_i+c_{i+1}\right)\]
        this should hold for all nodes $...,i-1, i, i+1,...$ \\
        reduce the index by 1
        \[b_{i-1}=\frac{f_i-f_{i-1}}{h_{i-1}}-\frac{h_{i-1}}{3}\left(2c_{i-1}+c_i\right)\]
        repeat for the result from the derivative condition to get
        \[b_i=b_{i+1}+h_{i-1}\left(c_{i-1}+c_i\right)\] 
				\btVFill
			\end{frame}	
	
      \begin{frame}
				\frametitle{\sectionIIIsubsectionIVtitle} 
				\bigskip
        Combine to find final equation
        \[\frac{f_{i+1}-{f_i}}{h_i}-\frac{h_i}{3}\left(2c_i+c_{i+1}\right)=\frac{f_i-f_{i-1}}{h_{i-1}}-\frac{h_{i-1}}{3}\left(2c_{i+1+c_i}\right)+h_{i-1}\left(c_{i-1}+c_i\right)\]
				\[h_{i-1}c_{i-1}+2c_i\left(h_i+h_{i-1}\right)c_i+h_ic_i=3\frac{f_{i+1}-f_i}{h_i}-3\frac{f_i-f_{i-1}}{h_{i-1}}\]
        The terms on the right hand side can be replaced with the finite difference equation
        \[f\left[x_i,x_j\right]=\frac{f_i-f_j}{x_i-x_j}\]
				\[h_{i-1}c_{i-1}+2c_i\left(h_i+h_{i-1}\right)c_i+h_ic_i=3\left(f\left[x_{i+1},x_i\right]-f\left[x_i,x_{i-1}\right]\right)\]
        \btVFill
			\end{frame}
      	
      \begin{frame}
				\frametitle{\sectionIIIsubsectionIVtitle} 
				\bigskip
        The two additional required conditions still need to be applied\vspc
        Set the second derivative to zero at both ends of the spline
        \[s_1\left(x_1\right)=0=2c_1+6d_1\left(x_1-x_1\right)\]
        \[c_1=0\]
        \[s_1\left(x_n\right)=0=2c_1+6d_1\left(x_n-x_n\right)\]
        \[c_n=0\]
        \btVFill
			\end{frame}

      	

\section{\sectionIVtitle}\label{sectionIV}

	% Section IV
		% section IV Outline
		\begin{frame}
			\large \textbf{Topic 3 - \sectionIVtitle} \vspace{3mm}\\

			\begin{itemize}
				\item \hyperlink{sectionIVsubsectionI}{\sectionIVsubsectionItitle} \vspc %  section IV subsection I
				\item \hyperlink{sectionIVsubsectionII}{\sectionIVsubsectionIItitle} \vspc % section IV subsection II
				\item \hyperlink{sectionIVsubsectionIII}{\sectionIVsubsectionIIItitle} \vspc % section IV subsection III
				\item \hyperlink{sectionIVsubsectionIV}{\sectionIVsubsectionIVtitle} \vspc % section IV subsection IV
			\end{itemize}

		\end{frame}

		% section IV subsection I
		\subsection{\sectionIVsubsectionItitle}\label{sectionIVsubsectionI}

			\begin{frame}
				\frametitle{\sectionIVsubsectionItitle}
				\bigskip

		
				\btVFill
			\end{frame}

		% section IV subsection II
		\subsection{\sectionIVsubsectionIItitle}\label{sectionIVsubsectionII}

			\begin{frame}
				\frametitle{\sectionIVsubsectionIItitle}
				\bigskip


				\btVFill
			\end{frame}

			\begin{frame}
				\frametitle{\sectionIVsubsectionIItitle}
				\bigskip


				\btVFill
			\end{frame}	


		% section IV subsection III
		\subsection{\sectionIVsubsectionIIItitle}\label{sectionIVsubsectionIII}

			\begin{frame}
				\frametitle{\sectionIVsubsectionIIItitle}
				\bigskip

				
				\btVFill
			\end{frame}	

			\begin{frame}
				\frametitle{\sectionIVsubsectionIIItitle}
				\bigskip

				
				\btVFill
			\end{frame}	

\end{document}





