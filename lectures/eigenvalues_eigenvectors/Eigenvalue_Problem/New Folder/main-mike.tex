%\documentclass{beamer}
\documentclass[handout]{beamer}
\usepackage{beamerthemesplit}
\usepackage{amsmath}
\usepackage{listings}
\beamertemplateballitem

\title{Solution of Matrix Eigenvalue Problem}
\author{Mike Renfro}
%\author{Gautam Chandekar}
\date{October 12, 2004}

\begin{document}

\lstset{language=MATLAB,basicstyle=\ttfamily\small,showstringspaces=false}

\frame{\titlepage}

\section*{Outlines}
\subsection*{Part I: Review of Previous Lecture}
\frame{
  \nameslide{outline}
  \frametitle{Review of Previous Lecture}
  \tableofcontents[part=1]
}

\subsection*{Part II: Solution of Matrix Eigenvalue Problem}
\frame{
  \frametitle{Solution of Matrix Eigenvalue Problem}
  \tableofcontents[part=2]
}

\part{Review of Previous Lecture}
\frame{\partpage}

\frame{
  \frametitle{Review of Previous Lecture}

  \begin{itemize}
  \item<+-| alert@+> Jacobi Iteration Method
  \item<+-| alert@+> Gauss-Seidel Iteration Method
  \item<+-| alert@+> Use of Software Packages
  \end{itemize}
}

\part{Solution of Matrix Eigenvalue Problem}
\frame{\partpage}

\section{Standard Matrix Eigenvalue Problem}
\subsection{Introduction}
\frame{
  \frametitle{Standard Matrix Eigenvalue Problem: Introduction}

  Consider a system of equations in algebraic form
  \begin{align*}
    (a_{11}-\lambda) x_1 + a_{12} x_2 + a_{13} x_3 + \cdots + a_{1n} x_n =& 0 \\
    a_{21} x_1 + (a_{22}-\lambda) x_2 + a_{23} x_3 + \cdots + a_{2n} x_n =& 0 \\
    \vdots{}& \\
    a_{n1} x_1 + a_{n2} x_2 + a_{n3} x_3 + \cdots + (a_{nn}-\lambda) x_n =& 0 \\
  \end{align*}
  This is not a normal system of linear algebraic equations we're used
  to. For one, there are $n$ equations, but $n+1$ unknowns (the $x_i$
  values, and also $\lambda$). This particular system of equations is
  known as \emph{the standard eigenvalue problem}.

}

\subsection{Other Forms of the Standard Eigenvalue Problem}
\frame{
  \frametitle{Form 1}
  
  The three forms shown are all algebraically equivalent. Any system
  of equations that can be expressed in these forms is a standard
  eigenvalue problem.

  \begin{displaymath}
    \left[ \begin{array}{cccc}
        a_{11}-\lambda & a_{12} & \cdots & a_{1n} \\
        a_{21} & a_{22}-\lambda & \cdots & a_{2n} \\
        \vdots & \vdots & \ddots & \vdots \\
        a_{n1} & a_{n2} & \cdots & a_{nn}-\lambda
      \end{array} \right]
    \left\{ \begin{array}{c}
        x_1 \\
        x_2 \\
        \vdots \\
        x_n
      \end{array} \right\} =
    \left\{ \begin{array}{c}
        0 \\
        0 \\
        \vdots \\
        0
      \end{array} \right\}
  \end{displaymath}
}
\frame{
  \frametitle{Form 2}
  \begin{displaymath}
    \left( \left[ \begin{array}{cccc}
          a_{11} & a_{12} & \cdots & a_{1n} \\
          a_{21} & a_{22} & \cdots & a_{2n} \\
          \vdots & \vdots & \ddots & \vdots \\
          a_{n1} & a_{n2} & \cdots & a_{nn}
        \end{array} \right] -
      \lambda \left[ \begin{array}{cccc}
          1 & 0 & \cdots & 0 \\
          0 & 1 & \cdots & 0 \\
          \vdots & \vdots & \ddots & \vdots \\
          0 & 0 & \cdots & 1
        \end{array} \right] \right)
    \left\{ \begin{array}{c}
        x_1 \\
        x_2 \\
        \vdots \\
        x_n
      \end{array} \right\} =
    \left\{ \begin{array}{c}
        0 \\
        0 \\
        \vdots \\
        0
      \end{array} \right\}
  \end{displaymath}
  \begin{displaymath}
    \left( \left[A\right] - \lambda \left[ I \right] \right) \left\{ x \right\} = \left\{ 0 \right\}
  \end{displaymath}
}
\frame{
  \frametitle{Form 3}
  \begin{displaymath}
    \left[ \begin{array}{cccc}
        a_{11} & a_{12} & \cdots & a_{1n} \\
        a_{21} & a_{22} & \cdots & a_{2n} \\
        \vdots & \vdots & \ddots & \vdots \\
        a_{n1} & a_{n2} & \cdots & a_{nn}
      \end{array} \right]
    \left\{ \begin{array}{c}
        x_1 \\
        x_2 \\
        \vdots \\
        x_n
      \end{array} \right\} =
    \lambda
    \left\{ \begin{array}{c}
        x_1 \\
        x_2 \\
        \vdots \\
        x_n
      \end{array} \right\}
  \end{displaymath}
  \begin{displaymath}
    \left[ A \right] \left\{ x \right\} = \lambda \left\{ x \right\}
  \end{displaymath}
}

\subsection{Solvability of the Standard Eigenvalue Problem}
\frame{
  \frametitle{Solvability of the Standard Eigenvalue Problem}
  Recall form 2 of the standard eigenvalue problem:
  \begin{displaymath}
    \left( \left[A\right] - \lambda \left[ I \right] \right) \left\{ x \right\} = \left\{ 0 \right\}
  \end{displaymath}
  This system of equations has a solution for values of $\lambda$ that
  cause the determinant of the coefficient matrix to equal 0, that is:
  \begin{displaymath}
    \left| \left[ A \right] - \lambda \left[ I \right] \right| = 0
  \end{displaymath}
}

\frame{
  \frametitle{Characteristic Equation}
  
  Expanding out all the terms of the previous determinant
  \begin{displaymath}
    \left| \begin{array}{cccc}
        a_{11}-\lambda & a_{12} & \cdots & a_{1n} \\
        a_{21} & a_{22}-\lambda & \cdots & a_{2n} \\
        \vdots & \vdots & \ddots & \vdots \\
        a_{n1} & a_{n2} & \cdots & a_{nn}-\lambda
      \end{array} \right| = 0
  \end{displaymath}
  yields a long polynomial in terms of $\lambda$. This polynomial will
  be $n$th order, and will therefore have $n$ roots, each of which may
  be real or complex.

}

\section{General Eigenvalue Problem}
\subsection{Introduction}
\frame{
  \frametitle{General Eigenvalue Problem: Introduction}
  
  Many physical systems do not automatically present themselves as a
  standard eigenvalue problem, even though they can be reformatted as
  a standard eigenvalue problem. The form of a \emph{general
    eigenvalue problem} is
  \begin{displaymath}
    \left[ A \right] \left\{ x \right\} = \lambda \left[ B \right] \left\{ x \right\}
  \end{displaymath}
  where $[A]$ and $[B]$ are symmetric matrices of size $n \times n$.

}

\subsection{General Eigenvalue Problem Example}
\frame{
  \frametitle{General Eigenvalue Problem Example}
  \begin{columns}
    \begin{column}{0.5\textwidth}
      A forging hammer of mass $m_2$ is mounted on a concrete
      foundation block of mass $m_1$. The stiffnesses of the springs
      underneath the forging hammer and the foundation block are given
      by $k_2$ and $k_1$, respectively.
    \end{column}
    \begin{column}{0.5\textwidth}
      \pgfimage[width=0.9\textwidth]{fig41a}
    \end{column}
  \end{columns}
  
}

\frame{
  \frametitle{General Eigenvalue Problem Example}

  \begin{columns}
    \begin{column}{0.5\textwidth}
      The system undergoes simple harmonic motion at one of its
      natural frequencies $\omega$. That is:
      \begin{displaymath}
        x_1(t) = \cos ( \omega t + \phi_1 )
      \end{displaymath}
      \begin{displaymath}
        x_2(t) = \cos ( \omega t + \phi_2 )
      \end{displaymath}
      \begin{displaymath}
        a_1(t) = - \omega^2 x_1(t)
      \end{displaymath}
      \begin{displaymath}
        a_2(t) = - \omega^2 x_2(t)
      \end{displaymath}
    \end{column}
    \begin{column}{0.5\textwidth}
      \pgfimage[width=0.9\textwidth]{fig41a}
    \end{column}
  \end{columns}

}

\frame{
  \frametitle{General Eigenvalue Problem Example}

  \begin{columns}
    \begin{column}{0.7\textwidth}
      Each mass in the system obeys Newton's second law of motion,
      that is:
      \begin{displaymath}
        \Sigma F = m a
      \end{displaymath}
      Forces on the foundation block:
      \begin{itemize}
      \item forces from the lower springs, which counteracts motion in
        the $x$ direction at an amount $-k_1 x_1$
      \item forces from the upper springs, which act according to the
        amount of relative displacement of the masses $m_1$ and $m_2$:
        $-k_2 (x_1 - x_2)$
      \end{itemize}
    \end{column}
    \begin{column}{0.3\textwidth}
      \pgfimage[width=0.9\textwidth]{fig41b}
    \end{column}
  \end{columns}

}

\frame{
  \frametitle{General Eigenvalue Problem Example}

  \begin{columns}
    \begin{column}{0.7\textwidth}
      The equilibrium equation for the foundation mass is then
      \begin{align*}
        \Sigma F &= m a \\
        -k_1 x_1 -k_2 (x_1 - x_2) &= m_1 a \\
        (-k_2 - k_1) x_1 + k_2 x_2 &= m_1 a \\
        (-k_2 - k_1) x_1 + k_2 x_2 &= -m_1 \omega^2 x_1 \\
        (k_1 + k_2) x_1 - k_2 x_2 &= m_1 \omega^2 x_1
      \end{align*}
    \end{column}
    \begin{column}{0.3\textwidth}
      \pgfimage[width=0.9\textwidth]{fig41b}
    \end{column}
  \end{columns}

}

\frame{
  \frametitle{General Eigenvalue Problem Example}
  \begin{columns}
    \begin{column}{0.7\textwidth}
      Similarly, the equilibrium equation for the forging hammer mass
      is
      \begin{align*}
        -k_2 x_1 + k_2 x_2 &= m_2 \omega^2 x_2
      \end{align*}
    \end{column}
    \begin{column}{0.3\textwidth}
      \pgfimage[width=0.9\textwidth]{fig41b}
    \end{column}
  \end{columns}
} 

\frame{
  \frametitle{General Eigenvalue Problem Example}

  So the two equations of motion are
  \begin{align*}
    (k_1 + k_2) x_1 - k_2 x_2 &= m_1 \omega^2 x_1 \\
    -k_2 x_1 + k_2 x_2 &= m_2 \omega^2 x_2
  \end{align*}
  or in matrix form
  \begin{displaymath}
    \left[ \begin{array}{cc}
        k_1 + k_2 & -k_2 \\
        -k_2 & k_2
      \end{array} \right]
    \left\{ \begin{array}{c}
        x_1 \\
        x_2
      \end{array} \right\} =
    \omega^2
    \left[ \begin{array}{cc}
        m_1 & 0 \\
        0 & m_2
      \end{array} \right]
    \left\{ \begin{array}{c}
        x_1 \\
        x_2
    \end{array} \right\}
  \end{displaymath}
  
  This is a general eigenvalue problem
  \begin{displaymath}
    [A]\{x\} = \lambda [B] \{x\}
  \end{displaymath}
  where $[A]$ is the spring matrix, $\{x\}$ is the vector of $x$
  values, $\lambda$ is $\omega^2$, and $[B]$ is the mass matrix.

}

\section{Eigenvalue Solutions in MATLAB}
\subsection{Standard Eigenvalue Problems}
\frame{
  \frametitle{Eigenvalue Solutions in MATLAB: Standard Problems}
  
  The design of a mechanical component requires that the maximum
  principal stress to be less than the material strength. For a
  component subjected to arbitrary loads, the principal stresses
  $\sigma$ are given by the solution of the equation
  \begin{displaymath}
    \left[ \begin{array}{ccc}
        \sigma_{xx} & \tau_{xy} & \tau_{xz} \\
        \tau_{xy} & \sigma_{yy} & \tau_{yz} \\
        \tau_{xz} & \tau_{yz} & \sigma_{zz}
    \end{array} \right]
    \left\{ \begin{array}{c}
        l_x \\
        l_y \\
        l_z
      \end{array} \right\} =
    \sigma
    \left\{ \begin{array}{c}
        l_x \\
        l_y \\
        l_z
      \end{array} \right\}
  \end{displaymath}
  where the $\sigma$ values represent normal stresses in the $x$, $y$,
  and $z$ directions, and the $\tau$ values represent shear stresses
  in the $xy$, $xz$, and $yz$ planes. The $l$ values represent
  direction cosines that define the principal planes on which the
  principal stress occurs.

}

\frame{
  \frametitle{Eigenvalue Solutions in MATLAB: Standard Problems}
  \begin{columns}
    \begin{column}{0.7\textwidth}
      Determine the principal stresses and principal planes in a
      machine component for the following stress condition
      \begin{displaymath}
        \left[ \begin{array}{ccc}
            \sigma_{xx} & \tau_{xy} & \tau_{xz} \\
            \tau_{xy} & \sigma_{yy} & \tau_{yz} \\
            \tau_{xz} & \tau_{yz} & \sigma_{zz}
          \end{array} \right] =
        \left[ \begin{array}{rrr}
            10 & 4 & -6 \\
            4 & -6 & 8 \\
            -6 & 8 & 14
          \end{array} \right] \text{~MPa}
      \end{displaymath}
    \end{column}
    \begin{column}{0.3\textwidth}
      \pgfimage[width=0.9\textwidth]{fig413}
    \end{column}
  \end{columns}

}

\frame[containsverbatim]{
  \frametitle{MATLAB Solution}
  \begin{lstlisting}
clear all
sigma=[10  4 -6
        4 -6  8
       -6  8 14];
[dirs,stresses]=eig(sigma);
% diag(A) extracts the elements of the
% [A] matrix along the diagonal
principalStressList=diag(stresses)
principalDirs=dirs
  \end{lstlisting}
}

\frame[containsverbatim]{
  \frametitle{MATLAB Solution}
  \begin{lstlisting}
>> rao_p431
principalStressList =
  -10.4828
    9.3181
   19.1647
principalDirs =
   -0.2792    0.8343   -0.4754
    0.8905    0.4102    0.1970
   -0.3594    0.3683    0.8574
  \end{lstlisting}
}

\subsection{General Eigenvalue Problems}
\frame{
  \frametitle{Eigenvalue Solutions in MATLAB: General Problems}
  \begin{columns}
    \begin{column}{0.5\textwidth}
      Solve the forging hammer problem for the following values:
      \begin{itemize}
      \item $m_1 = 20000 \text{~kg}$
      \item $m_2 = 5000 \text{~kg}$
      \item $k_1 = 1 \times 10^7 \text{~N/m}$
      \item $k_2 = 5 \times 10^6 \text{~N/m}$
      \end{itemize}
    \end{column}
    \begin{column}{0.5\textwidth}
      \pgfimage[width=0.9\textwidth]{fig41a}
    \end{column}
  \end{columns}
}

\frame{
  \frametitle{Solving Eigenvalue Problems in MATLAB}
  
  Solving this eigenvalue problem will yield 2 eigenvalues equal to
  the square of the system's natural frequencies, and 2 corresponding
  $x$ vector values that show the relative displacements of the $m_1$
  and $m_2$ masses at those frequencies.

}

\frame[containsverbatim]{
  \frametitle{MATLAB Solution (Part 1)}

  \begin{lstlisting}
clear all;
% Define spring constants and masses
% for hammer and foundation block
k1=1e7;
k2=5e6;
m1=20000;
m2=5000;

% Define system stiffness matrix
K=[k1+k2 -k2
     -k2  k2];
% Define system mass matrix
M=[m1  0
    0 m2];
  \end{lstlisting}

}

\frame[containsverbatim]{

  \frametitle{MATLAB Solution (Part 2)}

  \begin{lstlisting}
% Solve general eigenvalue problem
[X,Omega2]=eig(K,M);
% diag(A) extracts the elements of the
% [A] matrix along the diagonal
Omega=diag(sqrt(Omega2));
% Scale column 1 of the [X] matrix by
% the row 1, column 1 X value
X(:,1)=X(:,1)/X(1,1);
% Scale column 2 of the [X] matrix by
% the row 1, column 2 X value
X(:,2)=X(:,2)/X(1,2);

Omega
X
  \end{lstlisting}

}

\frame[containsverbatim]{

  \frametitle{MATLAB Solution (Results)}

  \begin{lstlisting}
>> rao_ex42
Omega =
   18.9634
   37.2879
X =
    1.0000    1.0000
    1.5616   -2.5616
  \end{lstlisting}
}
\section{Homework}

\frame[containsverbatim]{

  \frametitle{Homework}
  
  Continue working on the Gauss-Seidel and other homework problems
  already assigned.

}

\end{document}

%%% Local Variables:
%%% mode: latex
%%% TeX-command-default: "LaTeX PDF"
%%% TeX-master: t
%%% End:
