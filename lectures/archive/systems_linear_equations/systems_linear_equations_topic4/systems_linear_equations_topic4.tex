% !TEX TS-program = xelatex
% !TEX encoding = UTF-8 Unicode

% Lecture Template for ME3001-001-Tristan Hill - Spring 2017 - Fall 2017 - Fall 2020
% Mechanical Engineering Analysis with MATLAB
% Module 3 - Systems of Linear Equations 
% Topic 4 - Gaussian Elimination
  

\documentclass[fleqn]{beamer} % for presentation (has nav buttons at bottom)

\usepackage{/home/thill/Documents/lectures/analysis_lectures/analysis_lectures}

\newcommand{\MNUM}{3\hspace{2mm}} % Module number
\newcommand{\TNUM}{4\hspace{2mm}} % Topic number 
\newcommand{\moduletitle}{Systems of Linear Equations} % Titles and Stuff
\newcommand{\topictitle}{Gaussian Elimination} 

\newcommand{\sectiontitleI}{Various Row-Reduction Methods} % More Titles and Stuff
\newcommand{\sectiontitleII}{Gaussian Elimination Technique}
\newcommand{\sectiontitleIII}{A Generalized Algorithm}
\newcommand{\sectiontitleIV}{---}

\author{ME3001 - Mechanical Engineering Analysis} 
\title{Lecture Module - \moduletitle}
\date{Mechanical Engineering\vspc Tennessee Technological University}

\begin{document}
	
	\lstset{language=MATLAB,basicstyle=\ttfamily\small,showstringspaces=false}
	
	\frame{\titlepage \center\begin{framed}\Large \textbf{Topic \TNUM - \topictitle}\end{framed} \vspace{5mm}}
	
	% Section 0: Outline
	\frame{
		\large \textbf{Topic \TNUM - \topictitle} \vspace{3mm}\\
		
		\begin{itemize}
			
			\item \sectiontitleI    \vspc % Section I
			\item \sectiontitleII 	\vspc % Section II
			\item \sectiontitleIII 	\vspc %Section III
			\item \sectiontitleIV 	\vspc %Section IV
			
		\end{itemize}
		
	}

\section{\sectiontitleI}

\frame{ 
  \frametitle{\sectiontitleI}
  
  The Gaussian Elimination method has many variations.  You may have used a different version in linear algebra, but that is fine. This method in generalized so that is can be automated easily with a computer program.      
	
}

\section{\sectiontitleII}

\frame{ 
  \frametitle{\sectiontitleII}
  
  The Gaussian Elimination consists of two main steps. Some variations of the method combine the two steps into a single procedure. \vspace{10mm}
  
  \begin{enumerate}

    \item Forward Elimination of Unknowns
    
    \item Backwards Substitution   
  
  \end{enumerate} 
	
}

\frame{ \small
  \frametitle{\sectiontitleII}
  
\underline{{\bf Step 1:} Forward Elimination of Unknowns}
		
			\begin{itemize}
				\item  Eliminate $x_1$ from equations $2$ to $n$
					\begin{itemize}
						\item Eliminate $x_1$ from equation $2$
							\begin{itemize}
								\item define the eliminating factor $f_{21}$ as $a_{21}/a_{11}$
								\item redefine $a_{21}$ as $a_{21}-a_{11}*f_{21}$
								\item redefine $a_{22}$ as $a_{22}-a_{12}*f_{21}$ \\
								. . . 
								\item redifine $a_{2n}$ as $a_{2n}-a_{1n}*factor$ \\
							\end{itemize}
						\item Eliminate $x_1$ from equation $3$
							\begin{itemize}
								\item define the eliminating factor $f_{31}$ as $a_{31}/a_{11}$
								\item redefine $a_{31}$ as $a_{31}-a_{11}*f_{31}$
								\item redefine $a_{32}$ as $a_{32}-a_{12}*f_{31}$ \\
								 . . .
								\item redefine $a_{3n}$ as $a_{3n}-a_{1n}*f_{31}$ \\
							\end{itemize}	
					\end{itemize}
			\end{itemize}
	}	
   \frame{ \small
  \frametitle{\sectiontitleII}
  \begin{itemize}				
				\item   Eliminate $x_2$ from equations $3$ to $n$ \\
				\begin{itemize}
						\item Eliminate $x_2$ from equation $3$
					\begin{itemize}
						\item define the eliminating factor $f_{32}$ as $a_{32}/a_{22}$
						\item redefine $a_{32}$ as $a_{32}-a_{22}*f_{32}$
						\item redefine $a_{33}$ as $a_{33}-a_{23}*f_{32}$\\
						. . .
						\item redefine $a_{3n}$ as $a_{3n}-a_{2n}*f_{32}$ \\
						
					\end{itemize}
				
				\end{itemize}	
				. . . 
				%\renewcommand\labelitemiii{\textperiodcentered}
				\item   Eliminate $x_{n-1}$ from equation $n$ 

					\begin{itemize}
						\item define the eliminating factor $f_{n,n-1}$ as $a_{n,n-1}/a_{n-1,n-1}$
						\item redefine $a_{n,n-1}$ as $a_{n,n-1}-a_{n-1,n-1}*f_{n,n-1}$
						
					\end{itemize}
				
	\end{itemize}    
	
}

\frame{ \small
  \frametitle{\sectiontitleII}
  {\bf Step 2:} Backwards Subsitution
%		 \renewcommand\labelitemi{\textbullet}
% 		\renewcommand\labelitemii{\textendash}
% 		\renewcommand\labelitemiii{\textasteriskcentered}
% 		\renewcommand\labelitemiv{\textperiodcentered}
		\begin{itemize}
			\item Solve Equations $n$ through $1$ \\
				\begin{itemize}
				\item Solve for $x_n$ as $\frac{b_n}{a_{n,n}}$\\
			
				\item Solve for $x_{n-1}$ as $\frac{b_{n-1}-(a_{n-1,n}x_n)}{a_{n-1,n-1}}$ \\
					
				\item Solve for $x_{n-2}$ as $\frac{b_{n-2}-(a_{n-2,n-1}x_{n-1})-(a_{n-2,n}x_{n})}{a_{n-2,n-2}}$\\
					. \\ .\\ . \\					 
				\item Solve for $x_{1}$ as $\frac{b_{1}-(a_{12}x_{2})- . . . -(a_{1,n-1}x_{n-1})-(a_{1,n}x_{n})}{a_{1,1}}$\\	
							
				\end{itemize}
		\end{itemize}
  
  }


\section{\sectiontitleIII}

\frame{ 
  \frametitle{\sectiontitleIII}
\begin{multicols}{2}
	\underline{{\bf Step 1:} Forward Elimination} \vspace{2mm}\\ 

		{\it for} \color{mypurple}k \color{black} {\it from} 1 {\it to} \color{mygreen}n\color{black}-1 \vspace{1mm}
	
		\hspace{3mm} {\it for} \color{blue}i \color{black} {\it from} \color{mypurple}k\color{black}+1 {\it to} \color{mygreen}n\color{black} \vspace{1mm}
		
		\hspace{6mm} fact$=a_{\color{blue}i\color{black},\color{mypurple}k}/a_{\color{mypurple}k\color{black},\color{mypurple}k\color{black}}$ \vspace{1mm}

		\hspace{9mm} {\it for} \color{red}j \color{black} {\it from} \color{mypurple} k \color{black}  {\it to} \color{mygreen}n\color{black} \vspace{1mm}

		\hspace{12mm} $a_{\color{blue}i\color{black},\color{red}j}=a_{\color{blue}i\color{black},\color{red}j}-fact\times a_{\color{mypurple}k\color{black},\color{red}j}$ \vspace{1mm}
		
\hspace{10mm}{\it end} \vspace{1mm}

\hspace*{9mm}$b_{\color{blue}i\color{black}}=b_{\color{blue}i\color{black}}-fact\times b_{\color{mypurple}k\color{black}}$ \vspace{1mm}

\hspace*{6mm}{\it end} \vspace{1mm}

{\it end}\vspace{15mm}
		%\hspace*{20mm} \scalebox{1.5}{\color{blue} end \color{black} }\\
		%\hspace{0mm} \scalebox{1.5}{\color{blue} end \color{black}}\\ \\
		%\hspace{0mm} \scalebox{1.5}{\color{blue} end \color{black}}\\
		\underline{{\bf Step 2:} Backwards Substitution} \vspace{2mm}
		
				$x_{\color{mygreen}n\color{black}}=b_{\color{mygreen}n\color{black}}/a_{\color{mygreen}n\color{black},\color{mygreen}n\color{black}}$ \vspace{2mm}
				
		{\it for} \color{blue}i \color{black} {\it from} \color{mygreen}n\color{black}-1 {\it to} 1 \vspace{4mm}
	
		\hspace*{10mm}$x_{\color{blue}i}=(b_{\color{blue}i}-\sum\limits^n_{\color{red}j\color{black}=\color{blue}i\color{black}+1}\left( a_{\color{blue}i\color{black},\color{red}j}x_{\color{red}j}\right)) /a_{\color{blue}i\color{black},\color{blue}i\color{black}}$	\vspace{2mm}
		
		{\it end} \vspace{2mm}
		
		\end{multicols}
}

\end{document}

%\begin{document}
%
%\textbf{ \LARGE ME 3001 Lecture, Systems of Linear Equations} \\\\
%\textbf{ \LARGE The Gaussian Elimination Algorithm} \\
%
%
% \renewcommand\labelitemi{\textbullet}
% \renewcommand\labelitemii{\textendash}
% \renewcommand\labelitemiii{\textasteriskcentered}
% \renewcommand\labelitemiv{\textperiodcentered}
%\begin{description}
%
%
%\item \textbf{ Simple Example (3x3)}
%
%\newpage	
%	\item \textbf{\LARGE Gaussian Elimination}\\	
%	
%\newpage
%\Large
%\item \textbf{ This is a 2 part process}
%	
%	
%	\begin{itemize}
%		\item {\bf Step 1:} Forward Elimination of Unknowns
%		
%			\begin{itemize}
%				\item  Eliminate $x_1$ from equations $2$ to $n$
%					\begin{itemize}
%						\item Eliminate $x_1$ from equation $2$
%							\begin{itemize}
%								\item define the eliminating factor $f_{21}$ as $a_{21}/a_{11}$
%								\item redefine $a_{21}$ as $a_{21}-a_{11}*f_{21}$
%								\item redefine $a_{22}$ as $a_{22}-a_{12}*f_{21}$ \\
%								. . . 
%								\item redifine $a_{2n}$ as $a_{2n}-a_{1n}*factor$ \\
%							\end{itemize}
%						\item Eliminate $x_1$ from equation $3$
%							\begin{itemize}
%								\item define the eliminating factor $f_{31}$ as $a_{31}/a_{11}$
%								\item redefine $a_{31}$ as $a_{31}-a_{11}*f_{31}$
%								\item redefine $a_{32}$ as $a_{32}-a_{12}*f_{31}$ \\
%								 . . .
%								\item redefine $a_{3n}$ as $a_{3n}-a_{1n}*f_{31}$ \\
%							\end{itemize}	
%					\end{itemize}
%				\item   Eliminate $x_2$ from equations $3$ to $n$ \\
%				\begin{itemize}
%						\item Eliminate $x_2$ from equation $3$
%					\begin{itemize}
%						\item define the eliminating factor $f_{32}$ as $a_{32}/a_{22}$
%						\item redefine $a_{32}$ as $a_{32}-a_{22}*f_{32}$
%						\item redefine $a_{33}$ as $a_{33}-a_{23}*f_{32}$\\
%						. . .
%						\item redefine $a_{3n}$ as $a_{3n}-a_{2n}*f_{32}$ \\
%						
%					\end{itemize}
%				
%				\end{itemize}	
%				. . . 
%				\renewcommand\labelitemiii{\textperiodcentered}
%				\item   Eliminate $x_{n-1}$ from equation $n$ 
%
%					\begin{itemize}
%						\item define the eliminating factor $f_{n,n-1}$ as $a_{n,n-1}/a_{n-1,n-1}$
%						\item redefine $a_{n,n-1}$ as $a_{n,n-1}-a_{n-1,n-1}*f_{n,n-1}$\
%						
%					\end{itemize}
%				
%			\end{itemize}
%			\newpage
%		\item {\bf Step 2:} Backwards Subsitution
%		 \renewcommand\labelitemi{\textbullet}
% 		\renewcommand\labelitemii{\textendash}
% 		\renewcommand\labelitemiii{\textasteriskcentered}
% 		\renewcommand\labelitemiv{\textperiodcentered}
%		\begin{itemize}
%			\item Solve Equations $n$ through $1$ \\
%				\begin{itemize}
%				\item Solve for $x_n$ as \scalebox{1.3}{$\frac{b_n}{a_{n,n}}\]\\
%			
%				\item Solve for $x_{n-1}$ as \scalebox{1.3}{$\frac{b_{n-1}-(a_{n-1,n}x_n)}{a_{n-1,n-1}}\] \\
%					
%				\item Solve for $x_{n-2}$ as \scalebox{1.3}{$\frac{b_{n-2}-(a_{n-2,n-1}x_{n-1})-(a_{n-2,n}x_{n})}{a_{n-2,n-2}}\]\\
%					. \\ .\\ . \\					 
%				\item Solve for $x_{1}$ as \scalebox{1.3}{$\frac{b_{1}-(a_{12}x_{2})- . . . -(a_{1,n-1}x_{n-1})-(a_{1,n}x_{n})}{a_{1,1}}\]\\	\\\\
%							
%				\end{itemize}
%		\end{itemize}
%		\item Summary \\
%
%\newpage		
%		
%		\item {\bf The Forward Eimination Algorithm:}  \\
%
%		\scalebox{1.5}{ {\it for} \color{mypurple}k \color{black} {\it from} 1 {\it to} \color{mygreen}n\color{black}-1} \\\\
%	
%		\hspace{10mm} \scalebox{1.5}{{\it for} \color{blue}i \color{black} {\it from} \color{mypurple}k\color{black}+1 {\it to} \color{mygreen}n\color{black}} \\\\
%		
%		\hspace{20mm} \scalebox{1.5}{fact$=a_{\color{blue}i\color{black},\color{mypurple}k}/a_{\color{mypurple}k\color{black},\color{mypurple}k\color{black}}\] \\
%
%		\hspace{20mm} \scalebox{1.5}{{\it for} \color{red}j \color{black} {\it from} \color{mypurple} k \color{black}  {\it to} \color{mygreen}n\color{black}}\\\\
%
%		\hspace{30mm} \[a_{\color{blue}i\color{black},\color{red}j}=a_{\color{blue}i\color{black},\color{red}j}-$fact$\times a_{\color{mypurple}k\color{black},\color{red}j}\]\\\\
%\hspace*{20mm}\scalebox{1.5}{{\it end}}\\\\
%\hspace*{20mm}\[b_{\color{blue}i\color{black}}=b_{\color{blue}i\color{black}}-$fact$\times b_{\color{mypurple}k\color{black}}\]\\\\
%\hspace*{10mm}\scalebox{1.5}{{\it end}}\\ \\	
%\scalebox{1.5}{{\it end}}\\ 	 
%		%\hspace*{20mm} \scalebox{1.5}{\color{blue} end \color{black} }\\
%		%\hspace{0mm} \scalebox{1.5}{\color{blue} end \color{black}}\\ \\
%		%\hspace{0mm} \scalebox{1.5}{\color{blue} end \color{black}}\\
%
%		
%		\item {\bf The Backwards Substitution Algorithm:}  \\
%		
%		\[x_{\color{mygreen}n\color{black}}=b_{\color{mygreen}n\color{black}}/a_{\color{mygreen}n\color{black},\color{mygreen}n\color{black}}\]\\\\
%		\scalebox{1.5}{{\it for} \color{blue}i \color{black} {\it from} \color{mygreen}n\color{black}-1 {\it to} 1} \\\\
%	
%		\hspace{10mm}\[x_{\color{blue}i}=(b_{\color{blue}i}-$$\sum\limits^n_{\color{red}j\color{black}=\color{blue}i\color{black}+1}$$\left( a_{\color{blue}i\color{black},\color{red}j}x_{\color{red}j}\right)) /a_{\color{blue}i\color{black},\color{blue}i\color{black}}\]	\\\\
%		\scalebox{1.5}{{\it end}}\\ 
%
%\newpage	
%%		 \item A close look at backwards subsitution \\\\
%%			\hspace{10mm}\[x_{\color{blue}i}=(b_{\color{blue}i}-$$\sum\limits^n_{\color{red}j\color{black}=\color{blue}i\color{black}+1}$$\left( a_{\color{blue}i\color{black},\color{red}j}x_{\color{red}j}\right)) /a_{\color{blue}i\color{black},\color{blue}i\color{black}}\]	\\\\
%%		\scalebox{1.5}{{\it end}}\\ 	\\\\
%		
%		
%		 \newpage
%	
%		
%\end{itemize}
%		
%
%
%
%
%\newpage 
%
%	 \item \textbf{ \LARGE REMINDER - Homework 2 is due Wed. Sep. 26} \\
%	\item \textbf{ \LARGE REMINDER - MATLAB script from today's lecture will be posted on ilearn. } \\
%
%\end{description}
%
%
%	
%
%\end{document}
%
%
%
