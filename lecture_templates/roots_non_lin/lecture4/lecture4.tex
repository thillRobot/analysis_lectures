% Lecture Template for ME3001-001-Tristan Hill - Spring 2017
% 
% Mechanical Engineering Analysis with MATLAB
%
% Roots of Non- Linear Equations - Lecture 3
% MATLAB review

% Document settings
\documentclass[11pt]{article}
\usepackage[margin=1in]{geometry}
\usepackage[pdftex]{graphicx}
\usepackage{multirow}
\usepackage{setspace}
\usepackage{hyperref}
\usepackage{color,soul}
\usepackage{fancyvrb}
\usepackage{framed}
\usepackage{wasysym}
\usepackage{multicol}

\pagestyle{plain}
\setlength\parindent{0pt}
\hypersetup{
    bookmarks=true,         % show bookmarks bar?
    unicode=false,          % non-Latin characters in Acrobat’s bookmarks
    pdftoolbar=true,        % show Acrobat’s toolbar?
    pdfmenubar=true,        % show Acrobat’s menu?
    pdffitwindow=false,     % window fit to page when opened
    pdfstartview={FitH},    % fits the width of the page to the window
    pdftitle={My title},    % title
    pdfauthor={Author},     % author
    pdfsubject={Subject},   % subject of the document
    pdfcreator={Creator},   % creator of the document
    pdfproducer={Producer}, % producer of the document
    pdfkeywords={keyword1} {key2} {key3}, % list of keywords
    pdfnewwindow=true,      % links in new window
    colorlinks=true,       % false: boxed links; true: colored links
    linkcolor=red,          % color of internal links (change box color with linkbordercolor)
    citecolor=green,        % color of links to bibliography
    filecolor=magenta,      % color of file links
    urlcolor=blue           % color of external links
}

\definecolor{mygreen}{rgb}{0, .39, 0}

%\definecolor{dred}{#8B0000}
% [153,50,204] - dark orchid
\definecolor{mypurple}{rgb}{0.6,0.1961,0.8}
%[139,69,19] - saddle brown
\definecolor{mybrown}{rgb}{0.5451,0.2706,0.0745}

\definecolor{mygray}{rgb}{.6, .6, .6}

\setulcolor{red} 
\setstcolor{green} 
\sethlcolor{mygray} 

\newcommand{\VA}{\vspace{2mm}}
\newcommand{\VB}{\vspace{5mm}} 
 
\newcommand{\R}{\color{red}}
\newcommand{\B}{\color{blue}}
\newcommand{\K}{\color{black}}
\newcommand{\G}{\color{mygreen}}
\newcommand{\PR}{\color{mypurple}}

% assignment number 
\newcommand{\NUM}{2} 
\newcommand{\VSpaceSize}{2mm} 
\newcommand{\HSpaceSize}{2mm} 

\begin{document}

\textbf{ \LARGE ME 3001 Lecture, Roots of Non-Linear Equations} \\\\
\textbf{ \LARGE A Finite Difference Approach - The Secant Method } \\

\begin{itemize}
\Large
	\item \textbf{What does \PR{secant} \K mean?} \vspace{30mm}\\
	\item \textbf{The Newton-Raphson method is not \PR{purely numerical}\K, why?} \\\\
		\begin{itemize}
			\item  The Equation\vspace{30mm}	\\
			\item  The Derivation\vspace{30mm}	\\
		\end{itemize}
	\item \textbf{\LARGE  How can we avoid this issue?}\\\\

\newpage
	\item \textbf{ \LARGE Introduce the {\it Secant Method (modified Newton-Raphson)}}
\begin{itemize}	
	\item \LARGE{Forward Difference}\\\\
	\includegraphics[scale=.35]{lecture4_fig1.png}\\
	\item \LARGE{Backwards Difference}\\\\
	\includegraphics[scale=.35]{lecture4_fig1.png}\\
	\item \LARGE{Central Difference}\\\\
	\includegraphics[scale=.35]{lecture4_fig1.png}
\newpage
	\item \LARGE{These are know as {\it Finite Difference Approximations} .}\\
	\item \LARGE{When they are used in the {\it Newton-Raphson} equation this becomes the {\it Secant Method} .}\vspace{25mm}\\
	
	\item \LARGE{So what is different about using this method? }\\
		
\end{itemize}

		\newpage

\end{itemize}
\newpage 

% part 2 of the lecture
\textbf{ \LARGE A Brief Introduction to Optimization } \\

\begin{itemize}
\Large
	\item \textbf{\LARGE What is Optimization ?}
		\begin{itemize}
			\item Find Local Minima and Maxima \vspace{80mm}	
			\item Constraints	
		\end{itemize}
		\newpage
		
	\item \textbf{\LARGE Root finding and Optimization?}
		\begin{itemize}
			\item Using the derivative, 4$^{th}$ form of the problem... \vspace{80mm}	
	
		\end{itemize}
	\item \textbf{\LARGE What kind of problems do we solve? Think about the cone we designed a few days ago.}	\\
			\includegraphics[scale=.5]{lecture4_fig3.png}\hspace{5mm}
		 \scalebox{1.2}{surface area, $s=\pi r l = \pi r \sqrt{h^2+r^2}$} \\\\
		  \hspace*{60mm} \scalebox{1.2}{volume, $v=\pi r^2 \frac{h}{3}$} \\\\
		
		\newpage

	\item \textbf{\LARGE Optimization Techniques}\\\\
		\includegraphics[scale=1]{lecture4_fig2.png}
		\begin{itemize}
			\item Brute Force \vspace{30mm}
			\item Steepest Accent	
		\end{itemize}
		\newpage

\end{itemize}
\textbf{ \LARGE REMINDERS } \\

\begin{itemize}

	\item \textbf{ \LARGE Homework was due Friday but there is a late policy.} \\
	
	\item \textbf{ \LARGE The late policy has changed slightly. Please see the syllabus} \\
	 
	\item \textbf{ \LARGE MATLAB script from today's lecture will be posted on ilearn. } \\
		
 \end{itemize}




	

\end{document}





